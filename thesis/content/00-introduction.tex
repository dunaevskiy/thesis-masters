Architektura aplikací tvoří nedílnou součást každého vývoje.
Její náročnost je variabilní v závislosti na typu, složitosti, frekventovanosti použití a dalších parametrech aplikace.
Zpravidla čím menší nároky se na výsledek kladou, tím je navrhovaná architektura primitivnější, aby se ušetřilo na nákladech během vývoje a provozu aplikace.
Komplexnější požadavky však potřebují pro svůj plynulý provoz zajistit stabilní prostředí, které se definuje například stabilitou aplikace, způsobem opravy neočekávaných chyb, rychlosti odezvy apod.
Všechny tyto aspekty je mnohdy složité řešit najednou a přiřozeně se rozdělují na menší atomické celky, jež by se daly spravovat samostatně.

Koncept mikroservis\footnote{Též známý pod názvem mikroslužba} vznikl přibližně před 10 až 15 lety.
Explicitně, jako pojem, byl zmíněn až v roce 2011 v rámci popisu stylu architektury, se kterou se tehdy experimentovalo~\cite{msabegin}.
Prudký růst zájmu o \g{MSA} a mikroservisy celkově (dle statistických údajů Google Trends) byl zaznamenán v listopadu 2014 a začátku roku 2017.
V průběhu roku 2019 dosahoval největší popularity dle relativních počtů vyhledávání~\cite{googletrendsmsa}.
Aktuálně \g{MSA} zůstává velice frekventovaným trendem~\cite{googletrendsmsa}, z tohoto důvodu lze danou architekturu předpokládat za stále využívanou nebo alespon jevící zájem u určitého množství lidí.

Jelikož daný koncept je poměrné mladý, tak je málo pravděpodobné, že byl plně prozkoumán.
Představuje potenciální oblast pro experimentování a zdokonalování, modifikaci a odvozování jiných, nezávislých vývojových praktik.
Průzkum takových skutečností se nejlépe ověřuje v praxi, v daném případě budou vhodé středně velké aplikace s teoreticky neomezenou možností rozvoje.
Jako příklad takové struktury může být \g{IS} pro správu projektů v akademických institucích.

\clearpage



\section{Cíl práce}\label{sec:cil-prace}
Cílem této diplomové práce je primárně průzkum \g{MSA} se subjektivní úvahou o různých přístupech, technikách a myšlenkách týkajících se daného tématu.
Nabyté znalosti se uplatní v praxi ve formě \g{IS} pro správu projektů, který byl navržen a implementován jako prototyp monolitické architektury v bakalářské práci \uv{Informační systém pro správu studijních projektů}~\cite{bachelorthesis}.
I když tento systém bude brán jako stěžejní pro uplatnění znalostí o architektuře a bude uskutečněn přechod z monolitické architektury na architekturu mikroservis, tak za cíl není kladeno tento projekt výrazně zlepšit z uživatelského hlediska nebo uvést do provozu v době dokončení diplomové práce.
Veškerá analýza a specifikace uvedená v bakalářské práci~\cite{bachelorthesis}\TODO{kapitola} je z větší části považována za dostatečnou a splňující základní požadavky.
Některé aspekty však budou explicitně zaktualizovány a popsány v kapitolách \hyperref[ch:analysis]{Analýza předchozí práce} a \hyperref[ch:specification]{Specifikace}, protože mohou způsobovat nevhodný dopad -- bezpečnost, nevyhovující funkcionalita apod.
Vzhledem k potenciálně nekonečnému procesu rozvoje systému nebudou dodány různorodé interpretátory pro generování obsahu~\cite{bachelorthesis}\TODO{kapitola}, nýbrž pouze nejnutnější pro předvedení funkcionality.

Konečným výstupem bude především popis subjektivně prozkoumané oblasti \g{MSA} s případnými dodatečnými materiály, zrefaktorovaná a finalizovaná verze \g{IS}, jež je převedena z monolitické architektury na architekturu mikroservis, a dodané příručky uživatele a vývojáře.
V rámci implementace bude zdokonaleno testování a nasazování výsledné služby.


\clearpage



\section{Struktura}\label{sec:struktura}

\textbf{Kapitola 1} se věnuje stručné analýze výsledků předchozí práce, definuje silné a slabé stránky a navrhuje body pro zlepšení.

\textbf{Kapitola 2} se zabývá obecným zkoumáním problematiky \g{MSA} a typickými situacemi, které se potenciálně mohou řešit během implementace systému s využitím \g{MSA}.

\textbf{Kapitola 2} specifikuje požadavky nového projektu s návazností na analýzu předchozího projektu.
Nároky kladené na systém jsou uměle upraveny, aby směrovali na nutnost implementace \g{IS} v \g{MSA}.

\textbf{Kapitola 4} obsahuje analýzu a implementaci serverové části \g{IS} s uplatněním informací popsaných v kapitole \nameref{ch:msa}.

\textbf{Kapitola 5} obdobně, jako v případě předchozí kapitoly, obsahuje analýzu a implementaci klientské části aplikace.

\textbf{Kapitola 6} se věnuje zlepšení testování obou částí \g{IS}.
Na základě analýzy \g{MSA} jsou vybrány nástroje pro testování a je popsaná samotná realizace.

\textbf{Kapitola 7} se zabývá kontejnerizací a nasazováním \g{IS}.

Na závěr je zhodnocen výsledek a dosažení předem stanovených cílů a splnění zadání diplomové práce.

V přílohách a na přiloženém médiu jsou uvedeny zdrojové kódy obou částí \g{IS}, dokumentace pro vývojáře i pro uživatelé a jiné dodatečné materiály.

