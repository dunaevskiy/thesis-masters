\section{Cíl práce}\label{sec:cil-prace}
Cílem této diplomové práce je primárně průzkum \gls{MSA} se subjektivní úvahou o různých přístupech, technikách a myšlenkách týkajících se daného tématu.
Nabyté znalosti se uplatní v praxi ve formě \gls{IS} pro správu projektů, který byl navržen a implementován jako prototyp monolitické architektury v bakalářské práci \uv{Informační systém pro správu studijních projektů}~\cite{bachelorthesis}.
I když tento systém bude brán jako stěžejní pro uplatnění znalostí o architektuře a bude uskutečněn přechod z monolitické architektury na architekturu mikroservis, tak za cíl není kladeno tento projekt výrazně zlepšit z uživatelského hlediska nebo uvést do provozu v době dokončení diplomové práce.
Veškerá analýza a specifikace uvedená v bakalářské práci~\cite{bachelorthesis}\TODO{kapitola} je z větší části považována za dostatečnou a splňující základní požadavky.
Některé aspekty však budou explicitně zaktualizovány a popsány v kapitolách \hyperref[ch:analysis]{Analýza předchozí práce} a \hyperref[ch:specification]{Specifikace}, protože mohou způsobovat nevhodný dopad -- bezpečnost, nevyhovující funkcionalita apod.
Vzhledem k potenciálně nekonečnému procesu rozvoje systému nebudou dodány různorodé interpretátory pro generování obsahu~\cite{bachelorthesis}\TODO{kapitola}, nýbrž pouze nejnutnější pro předvedení funkcionality.

Konečným výstupem bude především popis subjektivně prozkoumané oblasti \gls{MSA} s případnými dodatečnými materiály, zrefaktorovaná a finalizovaná verze \gls{IS}, jež je převedena z monolitické architektury na architekturu mikroservis, a dodané příručky uživatele a vývojáře.
V rámci implementace bude zdokonaleno testování a nasazování výsledné služby.


\clearpage



\section{Struktura}\label{sec:struktura}

\textbf{Kapitola 1} se věnuje stručné analýze výsledků předchozí práce, definuje silné a slabé stránky a navrhuje body pro zlepšení.

\textbf{Kapitola 2} specifikuje požadavky nového projektu s návazností na analýzu předchozího projektu.

\textbf{Kapitola 3} se zabývá \gls{MSA}.

\textbf{Kapitola 4} obsahuje analýzu a implementaci serverové části aplikace.

\textbf{Kapitola 5} obsahuje analýzu a implementaci klientské části aplikace.

\textbf{Kapitola 6} se věnuje zlepšení testování obou částí \gls{IS}.

\textbf{Kapitola 7} se zabývá nasazováním \gls{IS}.

Na závěr je zhodnocen výsledek a dosažení předem stanovených cílů a splnění zadání diplomové práce.

V přílohách a na přiloženém médiu jsou uvedeny zdrojové kódy obou částí \gls{IS}, dokumentace pro vývojáře i pro uživatelé a jiné dodatečné materiály.

