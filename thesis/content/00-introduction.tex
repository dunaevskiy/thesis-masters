% FINAL, CHECKED

Architektura aplikací tvoří nedílnou součást každého vývoje.
Její náročnost je variabilní v závislosti na typu, složitosti, frekventovanosti použití a dalších parametrech aplikace.
Zpravidla čím menší nároky se na výsledek kladou, tím je navrhovaná architektura primitivnější, aby se ušetřilo na nákladech během vývoje a provozu aplikace.
Komplexnější požadavky však potřebují pro svůj provoz zajistit stabilní prostředí, které se definuje například stabilitou aplikace, způsobem opravy neočekávaných chyb, rychlosti odezvy apod.
Všechny tyto aspekty je mnohdy složité řešit najednou a přirozeně se rozdělují na menší atomické celky, jež by se daly spravovat samostatně.

Koncept mikroslužeb vznikl přibližně před 10 až 15 lety.
Explicitně, jako pojem, byl zmíněn až v roce 2011 v rámci popisu stylu architektury, se kterou se tehdy experimentovalo~\cite{msabegin}.
Prudký růst zájmu o \g{MSA} a mikroslužby celkově (dle statistických údajů Google Trends) byl zaznamenán v listopadu 2014 a začátku roku 2017.
V průběhu roku 2019 dosahoval největší popularity dle relativních počtů vyhledávání~\cite{googletrendsmsa}.
Aktuálně \g{MSA} zůstává velice frekventovaným trendem~\cite{googletrendsmsa}, z tohoto důvodu lze danou architekturu předpokládat za stále využívanou nebo alespoň jevící zájem u určitého množství lidí.

Jelikož daný koncept je poměrné mladý, tak je málo pravděpodobné, že byl plně prozkoumán.
Představuje potenciální oblast pro experimentování a zdokonalování, modifikaci a odvozování jiných, nezávislých vývojových praktik.
Průzkum takových skutečností se nejlépe ověřuje v praxi, v daném případě budou vhodné středně velké aplikace s teoreticky neomezenou možností rozvoje.
Jako příklad takové struktury může být \g{IS} pro správu projektů v akademických institucích.

\clearpage



\section{Cíl práce}\label{sec:cil-prace}
Cílem této diplomové práce je primárně průzkum \g{MSA} se subjektivní úvahou o různých přístupech, technikách a myšlenkách týkajících se daného tématu.
Samostatné části budou věnovány zpracováním databázových transakcí u serverové části a rozdělení klientské webové aplikace na menší, samostatné celky.

Nabyté znalosti se uplatní v praxi ve formě \g{IS} pro správu projektů, který byl původně navržen a implementován jako prototyp monolitické architektury v bakalářské práci \uv{Informační systém pro správu studijních projektů}~\cite{bachelorthesis}.
I když tento systém bude brán jako stěžejní pro uplatnění znalostí o architektuře a bude uskutečněn přechod z monolitické architektury na architekturu mikroslužeb, tak za cíl není kladeno projekt výrazně zlepšit z uživatelského hlediska nebo uvést do provozu v době dokončení diplomové práce.
Postupovat se bude inkrementálně – nejdřív se implementuje nejdůležitější, nezbytná osnova a zbylá funkcionalita bude dokončena dle časových možností, případně uvedena v seznamu pro další rozvoj.

Větší část analýzy a specifikace uvedené v bakalářské práci~\cite{bachelorthesis} bude převzata, některé aspekty však budou explicitně zaktualizovány a popsány v kapitolách \enquote{\nameref{ch:analysis}} a \enquote{\nameref{ch:spec}}, protože mohou způsobovat nevhodný dopad – bezpečnost, nevyhovující funkcionalita apod.

Konečným výstupem bude především popis subjektivně prozkoumané oblasti \g{MSA} sloužící jako rychlý přehled architektury mikroslužeb s případnými dodatečnými materiály, zrefaktorovaná verze \g{IS}, jež je převedena z monolitické architektury na architekturu mikroslužeb, a dodaná příručka vývojáře.
V rámci implementace bude zdokonaleno testování a nasazování výsledné služby.


\newpage



\section{Struktura}\label{sec:struktura}

\textbf{Kapitola 1} se věnuje stručné analýze výsledků předchozí práce, definuje silné a slabé stránky a navrhuje body pro zlepšení, zejména pro \g{MSA}.

\textbf{Kapitola 2} se zabývá obecným zkoumáním problematiky \g{MSA} a typickými situacemi, které se potenciálně mohou řešit během implementace systému s využitím \g{MSA}\\na~straně serveru i klienta.

\textbf{Kapitola 3} rozebírá integraci mikroslužeb s datovými úložišti a problematiku transakčního zpracování v případě oddělených úložišť.

\textbf{Kapitola 4} specifikuje požadavky nového projektu s návazností na analýzu předchozí práce a v souladu se zadáním diplomové práce.

\textbf{Kapitola 5} obsahuje analýzu a implementaci serverové části \g{IS} v \g{MSA} s uplatněním informací popsaných v kapitolách analýzy \g{MSA}.

\textbf{Kapitola 6} obdobně, jako v případě 5.
kapitoly, obsahuje analýzu a implementaci, ale již klientské části aplikace.

\textbf{Kapitola 7} se věnuje popisu zvloleného testování obou částí \g{IS} a automatizaci.

\textbf{Kapitola 8} se zabývá kontejnerizací a nasazováním \g{IS}.

\textbf{Kapitola 9} je věnována nedostatkům aktuálního řešení a potenciálnímu budoucímu rozvoji.

\textbf{Kapitola 10} porovnává původní monolitický prototyp s novou realizací systému a uvádí přínosy a nevýhody přechodu.

V závěrečné kapitole je zhodnocen výsledek, dosažení předem stanovených cílů a splnění zadání diplomové práce.

V přílohách a na přiloženém médiu jsou uvedeny zdrojové kódy všech částí \g{IS}, dokumentace pro vývojáře a jiné dodatečné materiály.
