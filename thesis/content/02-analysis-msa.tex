\chapter{Analýza mikroservisní architektury}\label{ch:msa}




Softwarová architektura jako pojem není přesně definována \cite{softarch}

Arichitektura může být popsána jako jistý řád a pravidla, dle kterých je celá aplikace postavena.
Existují nějaké šablony architektury, které jsou vhodné pro vývoj aplikací


V dané kapitole bude popisována především \gls{MSA} a bude porovnávana s jinými architekturami, které by ji potenciálně mohly nahradit.
Tyto vybrané architektury – \gls{MA}, \gls{SOA}\footnote{též architektura orientovaná na služby}, \gls{SA} – budou zkoumány vzhledem k přístupu k určitým aspektům, poskytovaným možnostem a výhodám a nevýhodám vůči \gls{MSA}.



Služba (bez vazby na architekturu) je chápána jako atomicky fungující celek, z větší části nezávislý na ostatních – soutředí se na konkrtétní funkcionalitu, má vlastní databázi (nebo schéma) s výhradným přístupem a veškerá komunikace probíhá přes striktně definované rozhraní.
Jakýkoliv jiný vliv, než přes dohodnuté rozhraní, musí být eliminován.

Jelikož existuje nespočetné množství modifikací výše uvedených architektur, bude se předpokládat, že se jedná o takto definované instance:


%https://www.ibm.com/cloud/learn/soa#toc-soa-vs-mic-BjTfju28

\begin{dl}
   \item[\gls{MA}] – jednoprocesový program atomické povahy – nelze z něho jednoduše vyčlenit funkční celky, které by se daly beze změn využívat v jiných programech.
   Obsahuje globální jednorázové připojení k databázi, které se provádí během startu programu a kontroluje aktuální stav migrací (případně vykonává chybějící).
   \item[\gls{SOA}] – jednoprocesový program s interním rozdělením založeným na \gls{ESB} – sběrnicí určenou pro centralizaci komunikace mezi službami.
   Jednotlivé služby tvoří samostatně fungující moduly, jež veškerou komunikaci provádí skrz přesně definované rozhraní \gls{ESB}.
   Obdobně, jako u \gls{MA} existuje jedno
   \item[\gls{MSA}] –
   \item[\gls{SA}] – architektura aplikace, která postrádá neustále běžící serverový proces a je pouze rozmístěna na \gls{FaaS} řešení.
   Jinými slovy funkcionalita není spuštěna v nepřerušovaném prostředí (jako démon), ale je dostupná na požádání.
   Při dotazu (například REST) je vytvořena potřebná instance aplikace, případně navázáno databázové spojení, vykonán požadavek a následně je instance odstraněna.
   Taková aplikace nemůže být stavová v obvyklém slova smyslu.
   Kvůli průběhu zpracování se rovněž nehodí pro náročné aplikace. \TODO{zdroj}
\end{dl}

Pro porovnání architektur bylo vybráno několik klíčových pojmů, které mohou během návrhem a implementací programu mít nějvětší vliv na rozhodování o výběru architektury.

+- MA OR 29


\begin{dl}
   \item[Datové úložiště] – využití persistentního úložiště (SQL/NoSQL databáze) pro zápis a čtení informací.
   Počáteční inicializace databáze, vytvoření struktury, migrace
\end{dl}
\begin{ul}
   \item \gls{MA} – není náročná pro testování.
   \item \gls{SOA} – rovněž může být testována
   \item \gls{MSA} – plnohodnotné testování celé struktury vyžaduje její nasazení na server.
\end{ul}

\begin{dl}
   \item[Nezávislost] – popis
\end{dl}
\begin{ul}
   \item \gls{MA} – něco umí
   \item \gls{SOA} – něco umí
   \item \gls{MSA} – něco umí
\end{ul}

\begin{dl}
   \item[Radikální změny] – složitost změny nebo přidání business logiky do aplikace, která by měl dopad na větší část dosavadní aplikace.
\end{dl}
\begin{ul}
   \item \gls{MA} –
   Z hlediska datového úložiště bude potřeba dělat pouze jednu migraci
   \item \gls{SOA} – něco umí
   \item \gls{MSA} – něco umí
\end{ul}

\begin{dl}
   \item[Tolerace chyb] – popis
\end{dl}
\begin{ul}
   \item \gls{MA} – něco umí
   \item \gls{SOA} – něco umí
   \item \gls{MSA} – něco umí
\end{ul}

\begin{dl}
   \item[Komunikační latence] – popis
\end{dl}
\begin{ul}
   \item \gls{MA} – něco umí
   \item \gls{SOA} – něco umí
   \item \gls{MSA} – něco umí
\end{ul}

\begin{dl}
   \item[Škálování] – popis
\end{dl}
\begin{ul}
   \item \gls{MA} – něco umí
   \item \gls{SOA} – něco umí
   \item \gls{MSA} – něco umí
\end{ul}

\begin{dl}
   \item[Konzistence] – popis
\end{dl}
\begin{ul}
   \item \gls{MA} – něco umí
   \item \gls{SOA} – něco umí
   \item \gls{MSA} – něco umí
\end{ul}

\begin{dl}
   \item[Nasazení] – popis
\end{dl}
\begin{ul}
   \item \gls{MA} – něco umí
   \item \gls{SOA} – něco umí
   \item \gls{MSA} – něco umí
\end{ul}

\begin{dl}
   \item[Testování] – schopnost psaní automatizovaných testů, které se vyplní při spuštění aplikace.
\end{dl}
\begin{ul}
   \item \gls{MA} – něco umí
   \item \gls{SOA} – něco umí
   \item \gls{MSA} – něco umí
\end{ul}

\section{Využití \g{MSA} pro modelování světa}\label{sec:msa-model-of-world}

Pro člověka nejspíš neexistuje nic přirozenějšího, než prostředí, ve kterém se pohybuje a kterému rozumí – od osobních věcí a proseců, jež musí vykonávat, až po uspořádání světa – město, stát, planeta, politické a sociální vztahy, komunikace s institucemi a interakce s rodinou.

Všechny tyto činnosti můžeme popsat pomocí subjektů – samostatných účastníků procesů, rozhraní, které poskytují pro komunikaci, a zpráv\footnote{Zprávou se rozumí informace poskytnutá v samostatné struktuře – například \g{JSON} nebo \g{XML}}, jež se předávají v rámci komunikace.
Každý subjekt je skupinou izolovaných funkcí s různě kompikovanou sadou komunikačních kanálů.
Může mít svoje potřeby a může vytvářet podněty pro ostatní subjekty.
Ne každý subjekt je schopen zpracovávat veškerou informaci, která k němu přichází od jiných subjektů.

Daný konecept naprosto přesně napodobuje \g{MSA}.
Jednotlivé subjekty jsou mikroslužby, mají vlastní rozhraní a vysílají informace v přesně definovaných strukturách.
Mohou existovat samostatně, i když jejich smysl existence nemusí existovat.
Zároveň jsou schopny zachytit a zpracovat zprávy, které byly určeny pro jejich použití a formát kterých je popsaný ve vnitřní logice.

V případě mírného zjednodušení se můžeme dostat ke konceptu \g{SOA}.
Subjekty zachovávají způsob komunikace s pomocí zpráv, ale již nejsou samostatní, potenciálně fungující celky, nýbrž moduly jednoho většího bloku.
Komunikace u takové architektury může být centrállně řízena \g{ESB}.\cite{soavsmsa}

Monolitní architektura v provonání se \g{SOA} zjednodušuje i samotné rozdělení do modulů.
Stále se jedná o samostatně fungující celek, ale vnitří struktura už postrádá moduly s odděleným rozhraním komunikace s využitím zpráv.
V rámci modelování světa se dá představit jako interně nedělitelný celek, který pouze poskytuje rozhraní pro komunikaci, tudíž je to subjekt a v rámci \g{MSA} může představovat mikroslužbu.

Na základě výše uvedených informací by logicky bylo nejjednodušší vždy vytvářet pouze \g{MSA} architekturu, protože je pro pochopení nejsnažší kvůli zkušenostem z každodenního života.
Každý subjekt má svoje potřeby (potřeva vytvořit zprávu, potřeba zpracovat příchozí právu) a o zbylé aspekty se nestará.
Problém nastává kvůli komplexnosti takového konceptu.
Jakékoliv rozdělení celku implicitně předpokládá nové náklady pro definování komunikace, které jsou náročné pro představu.
Proto může být nejvýhodnější začít architektorou s nejvíc redukovaným rozdělením – monolitem a dle potřeby měnit hloubku rozdělení.

Navíc při detailním zkoumání ve všech výše uvedených architekturách se dá vypozorovat rekurzivita.
Monolitická aplikace nebo \g{SOA} aplikace může tvořit mikroslužbu v rámci \g{MSA}.
\g{MSA} aplikace může tvořit mikroslužbu jiné \g{MSA} a fungovat jako monolit pro ostatní.




\section{Dekompozice na \gls{MSA}}

Na základě myšlenky o popisu světa budeme



Při modelování poskytnuté domény do \gls{MSA} je vhodné použít dekompozici.

\subsection{Způsoby vymezení služeb}

Business účel – doménu rozdělujeme dle business požadavků.

Subdomény – určujeme samostatné subdomény.


\section{Principy návrhu a návrhové vzory}

SOLID

High cohesion, low coupling



\section{Databáze jako zdroj dat pro \gls{MSA}}

\subsection{Relační databáze}


\subsection{NoSQL databáze}



\subsection{Agregace mezi schématy}
Jako jeden z možných pořadavků v rámci \gls{MSA} je agregace mezi schématy, což v rámci oddělených služeb znamená netriviální úkol.
Agregaci v tomto případě nemůžeme nechávat na databázi, protože může být úplně oddělená.


\section{Závislosti mezi mikroservisami a organizace}

Mikroservisy jsou vždy nezávislé z hlediska procesů, ale musí udržovat kompatibilní \gls{API} s ostatním prostředím.
Takové závislosti se nejlépe kontrolují verzováním jednotlivých mikroservis s pomocí sématického verzování.
Zde je třeba najít konkrétní nástroj pro tuto kontrolu.


\section{Testování \gls{MSA}}

fdf

\section{Nasazování \gls{MSA}}

\subsection{Správa zdrojového kódu}
Udržování zdrojových souborů jako monorepo / multirepo.

\section{Dokumentace \gls{MSA}}

sdfds
