\chapter{Analýza mikroservisní architektury}\label{ch:msa}


Pro porovnání architektur bylo vybráno několik klíčových pojmů, které mohou během návrhem a implementací programu mít nějvětší vliv na rozhodování o výběru architektury.


\begin{dl}
   \item[Jednoduchost vývoje] – náklady spojené s vývojem programu a začlenění nových vývojářů do týmu.
\end{dl}
\begin{ul}
   \item \g{MA} – jednoduchý počáteční vývoj kvůli jednodušší koncepci architektury, \g{IDE} jsou takovému vývoji přizpůsobeny.
   S rostoucím vývojovým týmem může být problém udržování konzistentní a stabilní aplikace.
   Počáteční náklady pro začlenění nového vývojáře v pozdějších fázích mohou být vysoké kvůli potřebě pochopení celého systému a často zastaralých technologiích.~\cite{msachris}
   \item \g{SOA} – rozdělení na služby přináší možnost rozdělit vývoj mezi několik nezávislých týmů, jež budou mít přesně definované rozhraní.
   V případě vužiívání společných prvků, například \g{ESB}, je třeba zajistit, aby se konvence nerozcházely.
   \item \g{MSA} – velice podobný vývoji, jako \g{SOA}, služby jsou však nezávislé i technologicky, mohou být realizovány v různých jazycích a prostředích.
   Může to být přínosem, jelikož můžeme volit technologie dle potřeb každé služby, ale zhorší se tím univerzalita týmu (ne každý člen týmu bude mít potřebné znalosti).
\end{ul}

\begin{dl}
   \item[Radikální změny] – složitost změny nebo přidání business logiky do aplikace, která by měl dopad na větší část dosavadní aplikace.
\end{dl}
\begin{ul}
   \item \g{MA} – veškěré změny se týkají jednoho jediného celku zdrojového kódu.
   Nevýhodou je možný nekontrolovaný dopad na celou aplikaci, protože nejsou striktně oddělené části projektu~\cite{msachris}.
   Takové změny lze řešit například vhodným dělením zdrojového kódu na balíčky nebo knihovny.
   \item \g{SOA}, \g{MSA} – v případě vhodně zvolené soudržnosti a provázanosti radikální změny funkcionality by byly zčásti omezené službou jako takovou, případně rozhraním komunikace.
\end{ul}

\begin{dl}
   \item[Tolerace chyb] – chování systému v případě výskytu neočekávané chyby nebo výjimky, která není zpracována manuálně či \g{JS} prostředím.
\end{dl}
\begin{ul}
   \item \g{MA}, \g{SOA} – jelikož se jedná o jednoprocesovou aplikaci, tak jakákoliv neošetřená chyba způsobí kolaps celého systému.
   \item \g{MSA} – několikaprocesové prostředí zajišťuje větší toleranci chyb, ale vždy záleží na ovlivněné části systému.
   V případě sekundární služba, která komunikuje například přes asynchronní broker zpráv, výpadek nebude mít stejně závažný dopad, jako v případě sehlání samotného brokeru nebo jedné z klíčových služeb (autorizace apod.).
\end{ul}

\begin{dl}
   \item[Komunikační latence] – běhěm komunikace mezi jednotlivými funkcionalitami aplikace může dojít k zpomalení vzhledem ke zvolenému přístupu.
\end{dl}
\begin{ul}
   \item \g{MA} – v monolitu můžeme předpokládat přímé volání potřebných metod, latence je zde potenciálně minimální.
   \item \g{SOA} – vzhledem k definovanému komunikačnímu kanálu, který vyžaduje výměnu zpráv, můžeme pocítat s časem potřebným pro vytvoření, odeslání a přijetí zprávy před vykonáním potřebné činnosti.
   \item \g{MSA} – obdobně, jako v případě \g{SOA}, ale samotné doručení zprávy může být ještě víc zpomaleno vybranou technologií, například komunikace se vzdáleným serverem nebo prostřednictvím vnějšího brokeru zpráv.
\end{ul}

\begin{dl}
   \item[Datové úložiště] – využití persistentního úložiště (SQL/NoSQL databáze) pro zápis a čtení informací.
   Počáteční inicializace databáze, vytvoření struktury, migrace.

   U všech zkoumaných architektur může být využito jak jedno, tak i víc datových úložišť.
   V případě oddělených služeb (zejména u \g{MSA}) je někdy vhodné použít samostatné, oddělené datové úložiště pro každou službu pro zachování větší nezávislosti.
   Taková struktura přináší komplikace z hlediska provádění transakcí prováděné přes několik datových úložišť a agregace, které vzhledem k fyzickému oddělení nelze provádět jedním \g{SQL} dotazem.
   Víc o tomto rozdělení bude zmíněno v kapitole \enquote{\nameref{ch:msa-data}}.
\end{dl}

\begin{dl}
   \item[Horizontální škálování] – škálování aplikace pro rozdělení zátěže na systém.
\end{dl}
\begin{ul}
   \item \g{MA}, \g{SOA} – jednoduché škálování – vzniká větší počet instancí aplikace, které mohou být umístěny za prvkem vyvažující zátěž~\cite{msachris}.
   \item \g{MSA} – vzhledem k samostatným procesům všech služeb škálování vyžaduje pokročilejší infrastrukturu a správu, avšak poskytuje i pokročilejší možnosti.
   V případě nerovnoměrné zátěže systému je možné tuto čast samostatně škálovat (protože se jedná o samostatný proces) a přizpůsobovat potřebám.
\end{ul}

\begin{dl}
   \item[Testování] – přizpůsobenost architektury psaní automatizovaných testů a jejich schopnost spouštět se automaticky v předem definované situaci (během vývoje, po nasazení apod.).
   Zaměření především na intergrační, funkční testy, testování výkonu a spolehlivosti.
\end{dl}
\begin{ul}
   \item \g{MA}, \g{SOA} – jedoprocesová aplikace může být testována jako celek.
   Aplikace třetích stran mohou být nahrazeny falešnými servery\footnote{anglicky rovněž mock-server} poskytujícími předpřipravenými, nebo prázdnými daty.
   \item \g{MSA} – každá služba je zde jako jedna aplikace \g{MA}, všechny ostatní služby jsou pro ní třetí stranou, která musí být během integračních testů nahrazována.
   Jakékoliv
   Z hlediska výkonu všas máme čitší prostředí a na jejich základě můžeme lépe zvolit vhodné škálování.
\end{ul}

\begin{dl}
   \item[Nasazování] – časová náročnost a komplexita rozmístění plně funkční aplikace na server.
\end{dl}
\begin{ul}
   \item \g{MA}, \g{SOA} – jednoprocesové aplikace mají jeden životní (nasazovací) cyklus.
   \item \g{MSA} – v případě samostatných služeb nasazení každé představuje vlastní životní cyklus, nejspíš budou potřeba některé synchronizované události (například spouštění některých služeb později, než jiných).
   Složitost nasazovacího cyklu se řídí složitostí architektury.
\end{ul}


\section{Využití MSA pro modelování světa}\label{sec:msa-model-of-world}

Pro člověka nejspíš neexistuje nic přirozenějšího, než prostředí, ve kterém se pohybuje a kterému rozumí – od osobních věcí a procesů, jež musí vykonávat, až po strukturu jeho okolí – město, stát, planeta, sociální vztahy, komunikace s lidmi.

Všechny tyto činnosti můžeme popsat pomocí subjektů – samostatných účastníků procesů, rozhraní, které poskytují pro komunikaci, a zpráv\footnote{Zprávou se rozumí informace poskytnutá v samostatné struktuře – například věta nebo formát typu \g{JSON}}, jež se předávají v rámci komunikace.
Každý subjekt je skupinou izolovaných funkcí s různě komplikovanou sadou komunikačních kanálů.
Může mít svoje potřeby a vytvářet podněty pro ostatní subjekty.
Ne každý subjekt je schopen zpracovávat veškerou informaci, která k němu přichází od jiných subjektů.

Daný koncept naprosto přesně napodobuje \g{MSA}.
Jednotlivé subjekty jsou mikroslužby, mají vlastní rozhraní a vysílají informace v přesně definovaných strukturách.
Mohou existovat samostatně, i když jejich smysl existence nemusí existovat.
Zároveň jsou schopny zachytit a zpracovat zprávy, které byly určeny pro jejich použití a formát kterých je popsaný ve vnitřní logice.

V případě mírného zjednodušení se můžeme dostat ke konceptu \g{SOA}.
Subjekty zachovávají způsob komunikace s pomocí zpráv, ale již nejsou samostatní, potenciálně fungující celky, nýbrž moduly jednoho většího bloku.
Komunikace u takové architektury může být centrálně řízena \g{ESB}~\cite{soavsmsa}.

Monolitní architektura v porovnání se \g{SOA} zjednodušuje i samotné rozdělení do modulů.
Stále se jedná o samostatně fungující celek, ale vnitřní struktura už postrádá moduly s odděleným rozhraním komunikace s využitím zpráv.
V rámci modelování světa se dá představit jako interně nedělitelný celek, který pouze poskytuje rozhraní pro~komunikaci, tudíž je to subjekt a v rámci \g{MSA} může představovat mikroslužbu.

Na základě výše uvedených informací by logicky bylo nejjednodušší vždy vytvářet pouze \g{MSA} architekturu, protože je pro pochopení nejsnazší kvůli zkušenostem z každodenního života.
Každý subjekt má svoje potřeby (potřeba vytvořit zprávu, potřeba zpracovat příchozí právu) a o zbylé aspekty se nestará.
Problém nastává kvůli komplexnosti takového konceptu.
Jakékoliv rozdělení celku implicitně předpokládá nové náklady pro definování komunikace, které jsou náročné pro představu.
Proto může být nejvýhodnější začínat s architekturou předpokládající nejvíc redukovaným rozdělením – monolitem – a dle potřeby měnit hloubku rozdělení.

Navíc při detailním zkoumání ve všech výše uvedených architekturách se dá vypozorovat rekurzivita.
Monolitická aplikace, nebo \g{SOA} aplikace může tvořit mikroslužbu v~rámci \g{MSA}.
\g{MSA} aplikace může tvořit mikroslužbu jiné \g{MSA} a fungovat jako monolit pro ostatní.




\section{Dekompozice na \gls{MSA}}

Na základě myšlenky o popisu světa budeme



Při modelování poskytnuté domény do \gls{MSA} je vhodné použít dekompozici.

\subsection{Způsoby vymezení služeb}

Business účel – doménu rozdělujeme dle business požadavků.

Subdomény – určujeme samostatné subdomény.


\section{Principy návrhu a návrhové vzory}

SOLID

High cohesion, low coupling



\section{Databáze jako zdroj dat pro \gls{MSA}}

\subsection{Relační databáze}


\subsection{NoSQL databáze}



\subsection{Agregace mezi schématy}
Jako jeden z možných pořadavků v rámci \gls{MSA} je agregace mezi schématy, což v rámci oddělených služeb znamená netriviální úkol.
Agregaci v tomto případě nemůžeme nechávat na databázi, protože může být úplně oddělená.


\section{Závislosti mezi mikroservisami a organizace}

Mikroservisy jsou vždy nezávislé z hlediska procesů, ale musí udržovat kompatibilní \gls{API} s ostatním prostředím.
Takové závislosti se nejlépe kontrolují verzováním jednotlivých mikroservis s pomocí sématického verzování.
Zde je třeba najít konkrétní nástroj pro tuto kontrolu.


\section{Testování \gls{MSA}}

fdf

\section{Nasazování \gls{MSA}}

\subsection{Správa zdrojového kódu}
Udržování zdrojových souborů jako monorepo / multirepo.

\section{Dokumentace \gls{MSA}}

sdfds
