\chapter{Obecný úvod do architektury mikroslužeb}\label{ch:msa-intro}

\chaptersummary{
   \begin{ul}
      \item obecné informace o architektuře a proč je důležitá,
      \item porovnání vybraných architektur – \g{MSA}, \g{MA} a \g{SOA},
      \item typy dekompozice specifikace na \g{MSA} architekturu,
      \item komunikace mezi mikroslužbami a řešení závislostí,
      \item testování a validace mikroslužeb,
      \item nasazení mikroslužeb na server a základní monitorování,
      \item krátký úvod do služeb na straně webového klienta.
   \end{ul}
}


Softwarovou architekturu jako pojem je těžké exaktně definovat, každý vývojář k ní může přistupovat a vnímat jinak.
Obecně může být popsána jako jistý řád a pravidla, vzniklá následkem mnoha rozhodnutí v průběhu analýzy a vývoje produktu.
Veškerý následující rozvoj by se měl striktně řídit těmito pravidly, aby umožnil vznik dlouhodobě udržovatelného výsledku~\cite{softarch}.

Vývoj softwaru bez architektury nebo s nepřesně definovanou osnovou může být přínosný v krátkodobé perspektivě nebo z hlediska šetření počátečních nákladů na čas a finanční složky~\cite{softarch}.
V případě dlouhodobého projektu to však může znamenat hromadění technického dluhu~\cite{archoworthit}.
Dle článku~\cite{archoworthit} přínos architektury lze vizuálně znázornit grafem~\ref{fig:architecture-line}, kde je uvedena kumulativní funkcionalita v závislosti čase spotřebovaným projektem.
Používání architektury ze začátku způsobuje zpomalení vývoje, ale od jisté hranice výhody se stává přínosnou a urychluje rozvoj.
Tento předpoklad však funguje pouze v případě, že se jedná o dobře zvolenou a popsanou architekturu (v textové nebo diagramové podobě), která má pozitivní vliv a je dostatečně ohebná~\cite{archoworthit}.


\begin{figure}[htbp]
   \centering
   \includegraphics[max width=\textwidth]{assets/architecture-line}
   \caption[Bod přínosu dodržování architektury]{Bod přínosu dodržování architektury~\cite{archoworthit}}\label{fig:architecture-line}
\end{figure}


Ačkoliv neexistuje přesná definice architektury obecně, existují detailněji popsané typy architektur, které jsou vhodné pro vývoj aplikací.
Jejich volba a případná adaptace vyžaduje pochopení konečného cíle požadovaného výsledku~\cite{softarch}.
V návaznosti na zadání diplomové práce v dané kapitole bude popisována především \g{MSA} a bude porovnávána s jinými architekturami, které by ji potenciálně mohly nahradit.
Tyto vybrané architektury – \g{MA}, \g{SOA}\footnote{též architektura orientovaná na služby}, \g{SA} – budou zkoumány vzhledem k přístupu k určitým aspektům, poskytovaným možnostem a výhodám a nevýhodám vůči \g{MSA}.
Vzhledem k dříve použitému jazyku TypeScript/JavaScript budou i srovnání zaměřeno na tento jazyk, případně Node.js prostředí.


Než se začne s konkrétním porovnáním, je třeba definovat jeden společný pojem všech 3 architektur – službu.
Služba v této práci je chápána jako atomicky fungující celek, z~větší části nezávislý na ostatních – soustředí se na konkrétní funkcionalitě (nebo skupině funkcionalit), má přístup k databázovému úložišti a veškerá komunikace probíhá přes striktně definované rozhraní (vizualizaci takové služby je možné vidět na~obrázku~\ref{fig:service-abstract}).
Může fungovat jako celek poskytující, přijímající a zpracovávající nebo předávající datové zprávy.
Jakýkoliv jiný vliv, než přes dohodnuté rozhraní, je ignorován a musí být ideálně eliminován.
Rozhraní takové služby může být popsáno s pomocí samostatné dokumentace nebo dokumentovaného kódu.


\begin{figure}[htbp]
   \centering
   \includegraphics[max width=\textwidth]{assets/service}
   \caption{Abstraktní znázornění služby}\label{fig:service-abstract}
\end{figure}


Kvůli nejednoznačnosti pojmu \enquote{architektura} nejspíš existuje nespočetné množství modifikací a adaptací výše uvedených architektur (\g{MA}, \g{SOA}, \g{MSA}, \g{SA}), proto se bude předpokládat, že se jedná o takto definované instance:

\begin{dl}
   \item[\g{MA}] – jednoprocesový program atomické povahy – nelze z něho jednoduše vyčlenit funkční celky, které by se daly beze změn využívat v jiných programech.
   Obsahuje globální jednorázové připojení k datovému zdroji, které se provádí během startu.
   Takový program je sám o sobě službou.
   \item[\g{SOA}] – jednoprocesový program s interním rozdělením na služby, které mezi sebou komunikují s pomocí zpráv přímo s využitím vyčleněného rozhraní nebo přes \g{ESB} – sběrnicí určenou pro centralizaci komunikace mezi službami.
   Vazba na datový zdroje může, ale nemusí být jedna (každá služba může mít samostatné připojení).
   \item[\g{MSA}] – několikaprocesový systém služeb (každá služba má právě jeden proces) organizovaný do větší interně kompatibilní struktury.
   \item[\g{SA}] – architektura aplikace, která postrádá kontinuálně běžící serverový proces a je pouze rozmístěna na \g{FaaS} řešení.
   Jinými slovy funkcionalita není spuštěna v nepřerušovaném prostředí (jako démon), ale je dostupná na požádání.
   Během uživatelského dotazu je vytvořena potřebná instance aplikace, případně navázáno databázové spojení, vykonán požadavek a následně je instance odstraněna.
\end{dl}

\newpage

Pro porovnání architektur bylo vybráno několik klíčových pojmů, které mohou během návrhem a implementací programu mít nějvětší vliv na rozhodování o výběru architektury.


\begin{dl}
   \item[Jednoduchost vývoje] – náklady spojené s vývojem programu a začlenění nových vývojářů do týmu.
\end{dl}
\begin{ul}
   \item \g{MA} – jednoduchý počáteční vývoj kvůli jednodušší koncepci architektury, \g{IDE} jsou takovému vývoji přizpůsobeny.
   S rostoucím vývojovým týmem může být problém udržování konzistentní a stabilní aplikace.
   Počáteční náklady pro začlenění nového vývojáře v pozdějších fázích mohou být vysoké kvůli potřebě pochopení celého systému a často zastaralých technologiích.~\cite{msachris}
   \item \g{SOA} – rozdělení na služby přináší možnost rozdělit vývoj mezi několik nezávislých týmů, jež budou mít přesně definované rozhraní.
   V případě vužiívání společných prvků, například \g{ESB}, je třeba zajistit, aby se konvence nerozcházely.
   \item \g{MSA} – velice podobný vývoji, jako \g{SOA}, služby jsou však nezávislé i technologicky, mohou být realizovány v různých jazycích a prostředích.
   Může to být přínosem, jelikož můžeme volit technologie dle potřeb každé služby, ale zhorší se tím univerzalita týmu (ne každý člen týmu bude mít potřebné znalosti).
\end{ul}

\begin{dl}
   \item[Radikální změny] – složitost změny nebo přidání business logiky do aplikace, která by měl dopad na větší část dosavadní aplikace.
\end{dl}
\begin{ul}
   \item \g{MA} – veškěré změny se týkají jednoho jediného celku zdrojového kódu.
   Nevýhodou je možný nekontrolovaný dopad na celou aplikaci, protože nejsou striktně oddělené části projektu~\cite{msachris}.
   Takové změny lze řešit například vhodným dělením zdrojového kódu na balíčky nebo knihovny.
   \item \g{SOA}, \g{MSA} – v případě vhodně zvolené soudržnosti a provázanosti radikální změny funkcionality by byly zčásti omezené službou jako takovou, případně rozhraním komunikace.
\end{ul}

\begin{dl}
   \item[Tolerace chyb] – chování systému v případě výskytu neočekávané chyby nebo výjimky, která není zpracována manuálně či \g{JS} prostředím.
\end{dl}
\begin{ul}
   \item \g{MA}, \g{SOA} – jelikož se jedná o jednoprocesovou aplikaci, tak jakákoliv neošetřená chyba způsobí kolaps celého systému.
   \item \g{MSA} – několikaprocesové prostředí zajišťuje větší toleranci chyb, ale vždy záleží na ovlivněné části systému.
   V případě sekundární služba, která komunikuje například přes asynchronní broker zpráv, výpadek nebude mít stejně závažný dopad, jako v případě sehlání samotného brokeru nebo jedné z klíčových služeb (autorizace apod.).
\end{ul}

\begin{dl}
   \item[Komunikační latence] – běhěm komunikace mezi jednotlivými funkcionalitami aplikace může dojít k zpomalení vzhledem ke zvolenému přístupu.
\end{dl}
\begin{ul}
   \item \g{MA} – v monolitu můžeme předpokládat přímé volání potřebných metod, latence je zde potenciálně minimální.
   \item \g{SOA} – vzhledem k definovanému komunikačnímu kanálu, který vyžaduje výměnu zpráv, můžeme pocítat s časem potřebným pro vytvoření, odeslání a přijetí zprávy před vykonáním potřebné činnosti.
   \item \g{MSA} – obdobně, jako v případě \g{SOA}, ale samotné doručení zprávy může být ještě víc zpomaleno vybranou technologií, například komunikace se vzdáleným serverem nebo prostřednictvím vnějšího brokeru zpráv.
\end{ul}

\begin{dl}
   \item[Datové úložiště] – využití persistentního úložiště (SQL/NoSQL databáze) pro zápis a čtení informací.
   Počáteční inicializace databáze, vytvoření struktury, migrace.

   U všech zkoumaných architektur může být využito jak jedno, tak i víc datových úložišť.
   V případě oddělených služeb (zejména u \g{MSA}) je někdy vhodné použít samostatné, oddělené datové úložiště pro každou službu pro zachování větší nezávislosti.
   Taková struktura přináší komplikace z hlediska provádění transakcí prováděné přes několik datových úložišť a agregace, které vzhledem k fyzickému oddělení nelze provádět jedním \g{SQL} dotazem.
   Víc o tomto rozdělení bude zmíněno v kapitole \enquote{\nameref{ch:msa-data}}.
\end{dl}

\begin{dl}
   \item[Horizontální škálování] – škálování aplikace pro rozdělení zátěže na systém.
\end{dl}
\begin{ul}
   \item \g{MA}, \g{SOA} – jednoduché škálování – vzniká větší počet instancí aplikace, které mohou být umístěny za prvkem vyvažující zátěž~\cite{msachris}.
   \item \g{MSA} – vzhledem k samostatným procesům všech služeb škálování vyžaduje pokročilejší infrastrukturu a správu, avšak poskytuje i pokročilejší možnosti.
   V případě nerovnoměrné zátěže systému je možné tuto čast samostatně škálovat (protože se jedná o samostatný proces) a přizpůsobovat potřebám.
\end{ul}

\begin{dl}
   \item[Testování] – přizpůsobenost architektury psaní automatizovaných testů a jejich schopnost spouštět se automaticky v předem definované situaci (během vývoje, po nasazení apod.).
   Zaměření především na intergrační, funkční testy, testování výkonu a spolehlivosti.
\end{dl}
\begin{ul}
   \item \g{MA}, \g{SOA} – jedoprocesová aplikace může být testována jako celek.
   Aplikace třetích stran mohou být nahrazeny falešnými servery\footnote{anglicky rovněž mock-server} poskytujícími předpřipravenými, nebo prázdnými daty.
   \item \g{MSA} – každá služba je zde jako jedna aplikace \g{MA}, všechny ostatní služby jsou pro ní třetí stranou, která musí být během integračních testů nahrazována.
   Jakékoliv
   Z hlediska výkonu všas máme čitší prostředí a na jejich základě můžeme lépe zvolit vhodné škálování.
\end{ul}

\begin{dl}
   \item[Nasazování] – časová náročnost a komplexita rozmístění plně funkční aplikace na server.
\end{dl}
\begin{ul}
   \item \g{MA}, \g{SOA} – jednoprocesové aplikace mají jeden životní (nasazovací) cyklus.
   \item \g{MSA} – v případě samostatných služeb nasazení každé představuje vlastní životní cyklus, nejspíš budou potřeba některé synchronizované události (například spouštění některých služeb později, než jiných).
   Složitost nasazovacího cyklu se řídí složitostí architektury.
\end{ul}


\section{Využití MSA pro modelování světa}\label{sec:msa-model-of-world}

Pro člověka nejspíš neexistuje nic přirozenějšího, než prostředí, ve kterém se pohybuje a kterému rozumí – od osobních věcí a procesů, jež musí vykonávat, až po strukturu jeho okolí – město, stát, planeta, sociální vztahy, komunikace s lidmi.

Všechny tyto činnosti můžeme popsat pomocí subjektů – samostatných účastníků procesů, rozhraní, které poskytují pro komunikaci, a zpráv\footnote{Zprávou se rozumí informace poskytnutá v samostatné struktuře – například věta nebo formát typu \g{JSON}}, jež se předávají v rámci komunikace.
Každý subjekt je skupinou izolovaných funkcí s různě komplikovanou sadou komunikačních kanálů.
Může mít svoje potřeby a vytvářet podněty pro ostatní subjekty.
Ne každý subjekt je schopen zpracovávat veškerou informaci, která k němu přichází od jiných subjektů.

Daný koncept naprosto přesně napodobuje \g{MSA}.
Jednotlivé subjekty jsou mikroslužby, mají vlastní rozhraní a vysílají informace v přesně definovaných strukturách.
Mohou existovat samostatně, i když jejich smysl existence nemusí existovat.
Zároveň jsou schopny zachytit a zpracovat zprávy, které byly určeny pro jejich použití a formát kterých je popsaný ve vnitřní logice.

V případě mírného zjednodušení se můžeme dostat ke konceptu \g{SOA}.
Subjekty zachovávají způsob komunikace s pomocí zpráv, ale již nejsou samostatní, potenciálně fungující celky, nýbrž moduly jednoho většího bloku.
Komunikace u takové architektury může být centrálně řízena \g{ESB}~\cite{soavsmsa}.

Monolitní architektura v porovnání se \g{SOA} zjednodušuje i samotné rozdělení do modulů.
Stále se jedná o samostatně fungující celek, ale vnitřní struktura už postrádá moduly s odděleným rozhraním komunikace s využitím zpráv.
V rámci modelování světa se dá představit jako interně nedělitelný celek, který pouze poskytuje rozhraní pro~komunikaci, tudíž je to subjekt a v rámci \g{MSA} může představovat mikroslužbu.

Na základě výše uvedených informací by logicky bylo nejjednodušší vždy vytvářet pouze \g{MSA} architekturu, protože je pro pochopení nejsnazší kvůli zkušenostem z každodenního života.
Každý subjekt má svoje potřeby (potřeba vytvořit zprávu, potřeba zpracovat příchozí právu) a o zbylé aspekty se nestará.
Problém nastává kvůli komplexnosti takového konceptu.
Jakékoliv rozdělení celku implicitně předpokládá nové náklady pro definování komunikace, které jsou náročné pro představu.
Proto může být nejvýhodnější začínat s architekturou předpokládající nejvíc redukovaným rozdělením – monolitem – a dle potřeby měnit hloubku rozdělení.

Navíc při detailním zkoumání ve všech výše uvedených architekturách se dá vypozorovat rekurzivita.
Monolitická aplikace, nebo \g{SOA} aplikace může tvořit mikroslužbu v~rámci \g{MSA}.
\g{MSA} aplikace může tvořit mikroslužbu jiné \g{MSA} a fungovat jako monolit pro ostatní.

\section{Dekompozice na \g{MSA}}\label{sec:msa-decomposition}

Projekt, u něhož bylo rozhodnuto o \g{MSA}, vyžaduje, jako jeden z kroků před implementací, dekompozici zadání na oblasti, dle kterých budou následně vytvářeny jednotlivé služby.
Stejně jako v případě myšlenky modelování světa, účastníky procesů budou subjekty a jejich rozhraní pro komunikaci, které v případě projektu lze vyčíst specifikace.

Jinými slovy definování konkrétní architektury projektu se dá provádět postupně ve 3 fázích~\cite{msachris} (vizualizace na obrázku~\ref{fig:msa-decomposition-flow}):


\begin{ol}
   \item Definování operací zpracovávaných serverem – na základě specifikace projektu, ve které jsou popsány požadavky očekávané od serverové části aplikace, formulujeme do konkrétních volání – získat, vytvořit, aktualizovat data.
   \item Definování možných služeb – pro dekompozici na konkrétní služby existují dva základní přístupy – rozdělení dle subdomén a prozdělení dle obchodních potřeb~\cite{msachris}, oba přístupy budou stručně popsány v dalších podkapitolách.
   Oba přístupy vychází z navrhnutých volání a přiřazuje je jednotlivým službám.
   \item Definování propojení požadavků se službami a komunikace služeb mezi sebou – prohlubuje definovanou komunikaci mezi službami a vytváří konkrétní interní a externí spoje.
\end{ol}


\begin{figure}[htbp]
   \centering
   \includegraphics[max width=\textwidth]{assets/draft-decomposition-flow}
   \caption{Tři kroky návrhu \g{MSA}}\label{fig:msa-decomposition-flow}
\end{figure}


V rámci rozdělování operací do konkrétních služeb (stejně jako návrh služeb) mohou pomoci i návrhové vzory týkající se vývoje samotného, jako jsou například:


\begin{dl}
   \item[Princip jedné odpovědnosti] – každá třída musí mít právě jeden důvod, proč se měnit~\cite{srp}.
   V kontextu mikroslužeb se to rovněž vztahuje i na službu samotnou – musí se zabývat pouze jednou subdoménou řešeného problému.
   \item[Vysoká soudržnost] – služba obashuje vše potřebné pro řešení oblasti, za níž je odpovědná~\cite{lchc}.
   \item[Nízká provázanost] – služba může získávat informace z ostatních zdrojů, ale snížená provázanost ulehčuje vývoj~\cite{lchc}.
\end{dl}


Vzhledem k širokému rozsahu dostupných modifikací \g{MSA} je sem možné začlenit mnohem větší spektrum pravidel a doporučení, vždy záleží na aspektech konkrétního zadání.
Některé z nich budou popsány v následujících kapitolách a mohou mít vliv na rozhodnutí spojená s dekompozicí vytvářené funkcionality.



\subsection{Dekompozice dle obchodních potřeb}\label{subsec:msa-decomposition-business}
Dekompozice oblastí dle obchodních potřeb je jednou ze dvou základních možností, jak dokompozici provádět~\cite{msachris}.
Soutředí se na vazbě s architekturou společnosti, neboli něčem, co firmě přináší užitečnou hodnotu~\cite{decompbusiness}.
Uvažujme případ, že o vytvoření \g{MSA} požádala firma, která zpracovává různé druhý objednávek - oblečení, potraviny, sportovní nářadí a spravuje interní pracovníky – stálé zaměstnance a příležitostnou výpomoc.
Z takové struktury by mohlo vzniknout 5 služeb, které by spravovaly:


\begin{ul}
   \item objednávky oblečení,
   \item objednávky potravin,
   \item objednávky sportovního nářadí,
   \item správu zaměstnanců,
   \item správu výpomoci.
\end{ul}



\subsection{Dekompozice dle subdomén}\label{subsec:msa-decomposition-domain}
Dekompozice z hlediska subdomény, která představuje druhý způsob rozdělení, na rozdíl od obchodních požadavků nevnímá architekturu společnosti jako zásadní, i když je nutná, a upřednostňuje logické rozdělování~\cite{decompsubdomain}.
V případě stejného příkladu firmy by dekompozice dle subdomén mohla vypadat následovně:


\begin{ul}
   \item objednávky,
   \item správa pracovníků.
\end{ul}


Takové služby už by nebyly specializované na konkrétním užití a tudíž by mohly být větší nároky na jejich schopnost se přizpůsobovat potřebám \g{IS}.

Typ dekompozice nelze přesně stanovit, je třeba se dívat na všechny potřebné aspekty zkoumané domény a přizpůsobovat dané metody konkrétním situacím~\cite{msachris}.

\section{Komunikace \g{MS}}\label{sec:msa-communication}

Mikroslužby jsou definované jako vždy oddělené z hlediska procesů a pro komunikaci s jinými službami nebo externími systémy musí udržovat kompatibilní \g{API} vůči jejich rozhraní a prostředí.
U takové komunikace můžeme zkoumat typ, technologii a hierarchii/strukturu, jež se řídí požadavkami na systém.



\subsection{Typ komunikace}\label{subsec:msa-communication-type}
Typ komunikace se abstrahuje od konkrétních řešení a zkoumá pouze konceptuální potřeby komunikace systému.
Obecně se uvádí dělení na dvě podskupiny – dle počtu cílových služeb a dle synchronity komunikace, které se dají různě kombinovat, přícemž každá kombinace může mít víc podtypů~\cite{msachris}.
Celý přehled je uveden v následujícím seznamu:


\begin{dl}
   \item [Synchronní 1:1] – synchronní komunikace dvou subjektů.

   \begin{ul}
      \item \textbf{Dotaz/odpověď} – producent vytváří právě jeden dotaz na jinou službu a čeká na pozitivní, nebo negativní odpověď.
      Tento způsob vede ke zvýšení provázanosti \g{MS}.
   \end{ul}

   \item [Synchronní 1:N] – synchronní komunikace s více cíli nemůže existovat, protože by vznikla sekvence synchronních požadavků 1:1 kvůli jednomu producentu.

   \item [Asynchronní 1:1] – asynchronní komunikace dvou subjektů.

   \begin{ul}
      \item \textbf{Asynchronní dotaz/odpověď} – producent vytváří jeden dotaz a nečeká na okamžitou odpověď.
      Odpověď se může poskytnout kdykoliv (nebo vůbec) – chování nesmí být blokujícího typu.
      \item \textbf{Oznámení} – producent vytváří jeden dotaz (oznámení), zpětná komunikace se neočekává a neexistuje.
   \end{ul}

   \item [Asynchronní 1:N] – asynchronní komunikace s více cíli.

   \begin{ul}
      \item \textbf{Producent/konzument} – producent vytváří jeden dotaz a předává všem konzumentům.
      \item \textbf{Producent / asynchronní odpovědi} – producent vytváří jeden dotaz a poředává všem konzumentům.
      Následně čeká předem danou dobu na potenciální asynchronní odpovědi od konzumentů.
   \end{ul}
\end{dl}



\subsection{Technologie komunikace}\label{subsec:msa-communication-technology}
Technologií může být myšlen protokol nebo systém doporučení, která mohou být uplatněna v konkrétním programovacím jazyce.
Vzhledem k Node.js se může jednat například o následující způsoby:

\begin{dl}
   \item [REST] – běžné GET/POST/\ldots dotazy přes \g{HTTP}.
   \item [GraphQL] – dotazovací jazyk nad strukturovanými daty~\cite{graphql}.
   \item [gRPC] – framework založený na \g{RPC} protokolu~\cite{grpc}.
   \item [RabbitMq] – broker zpráv~\cite{rabbitmq}.
   \item [Kafka] – platforma pro distribuované streamování událostí~\cite{kafka}.
   \item [MQTT] – standard pro \g{IoT} komunikace~\cite{mqtt}.
\end{dl}

Výběr technologie má dopad na provázanost služeb v systému.
V případě například \g{REST} implementace bude provázanost větší kvůli požadovné odpovědi v relativně krátké době.
Volba RabbitMq naopak může zajistit menší provázanost tím, že zpracování se bude řídit brokerem zpráv, avšak v případě nutné okamžité odpovědi může způsobovat komplikace.


\subsection{Struktura komunikace}\label{subsec:msa-communication-structure}

Pokud pomineme konkrétní způsob výměny informací mezi mikroslužbami a podíváme na se obecnou organizaci komunikace, tak existují dva základní způsoby – orchestrace a choreografie~\cite{choreovsorch}.


\begin{dl}
   \item [Orchestrace] – předpokládá analogii s hudebním orchestrem, kdy každý člen systému zná vlastní roli, ale stejně musí být řízen dirigentem.
   V tomto případě dirigent je zastoupen orchestrační mikroslužbou, která vystupuje jako prvek, jenž zprostředkovává veškerou komunikaci.
   Tento koncept je úzce spjat s RESTful \g{API} a může v případě obrovského počtu \g{MS} způsobovat komplikace, protože orchestrační prvek bude muset zvládat stovky až tisíce mikroslužeb a ve výsledku dopadne jako těžce spravovatelná monolitická část~\cite{choreovsorch}.
   Z implementačního pohledu však může být o něco jednodušší, protože jakákoliv propagace chyb, sběr dat a další požadavky mohou být přímočářejší a soustředěny v jednom místě.

   \item [Choreografie] – analogicky představuje taneční skupinu, kde neexistuje řídící prvek, ale každý člen reaguje na podnět (hudbu) a provádí potřebné kroky.
   Implementacně musí existovat komunikace řízená událostmi, kdy jedna služba vysílá zprávu do brokera a o zbytek se nemusí starat, vše je řízeno asynchronně.
   Zárověň každá služba odposlouchává pouze zprávy, na které má reágovat.
   Tento koncept zajišťuje mnohem větší ohebnost a menší provázanost služeb~\cite{choreovsorch}.
   Přináší však potřebu lépe zpracovávat chyby kvůli asynchronním akcím, které nemusí nikdy doběhnout.
   Zejména se to projevuje u původně sycnhronních požadavků, například ze strany prohlížeče, kdy se musí čekat na odpověd.
   Aby nenastalo potenciálně nekonečné očekávání dokončení požadavku na server, tak se může zvolit přiměřeně dlouhá časová doba, kdy na straně serveru se vykonávání požadavku začne brát jako nezdařilé.
   V tomto případě se vrací chybná odpověd a je potřeba zrušit všechny změny, které byly provedené dosud a které se mohou provést po dokončení pomalu zpracovaného požadavku.
\end{dl}



\begin{figure}[htbp]
   \centering
   \includegraphics[max width=\textwidth]{assets/draft-msa-communication}
   \caption{Orchestrace a choreografie \g{MS}}\label{fig:msa-structure-communication}
\end{figure}



\section{Udržování závislostí}\label{sec:msa-dependencies}

V průběhu vývoje systému s \g{MSA} je pravděpodobné, že vlivem měnících se podmínek bude docházet i ke změnám potřeb poskytovaných rozhraní.
Takové modifikace mohou vést k nekonzistenci a narušení systému jako celku.
Tuto situaci je možné kontrolovat a řešit s pomocí sémantického verzování v různých variantách.


\subsection{Verzování v \g{URI}}\label{subsec:msa-dependencies-uri}

Zápis verze v \g{URI} může vypadat následovně:

\h{schema://domain/name-service/v1.0.0/entity}

Výhody:
\begin{ul}
   \item Je možné udržovat víc aktivních \g{API} verzí v rámci jedné \g{MS}.
   \item
\end{ul}

Nevýhody:
\begin{ul}
   \item V případě pevné vazby na verzi pro klientskou aplikaci může být nákladné sledovat aktualizace \h{minor} nebo \h{patch} verzí.
   Toto může být řešeno redukcí pouze na \h{major} verze (\h{.../v1/...}).
\end{ul}


\subsection{Verzování v hlavičkách \g{HTTP}}\label{subsec:msa-dependencies-headers}



\subsection{Verzování \g{MSA}}\label{subsec:msa-dependencies-msa}
Místo \g{API} je možné verzovat a hlídat samotné mikroslužby.
Výrazně se tím zkomplikuje možnost udržování několika poskytovaných rozhraní, protože každá verze bude vyžadovat vlastní adresu v rámci systému.
Na druhou stranu je možné u každé sluby definovat seznam závislostí, který by se poskytoval v rámci obecného dotazu ohledně stavu služby.
V případě existence konceptu registrace mikroslužby v systému by bylo možné tyto požadavky na závislosti kontrolovat a včas upozorňovat na nevhodnout sestavu mikroslužeb (viz obrázek~\ref{fig:version-reg}).


\begin{figure}[htbp]
   \centering
   \includegraphics[max width=\textwidth]{assets/draft-version-reg}
   \caption{Registrace \g{MS} a kontrola závislostí}\label{fig:version-reg}
\end{figure}

\section{Testování a automatizace}\label{sec:testing}

Daná kapitola se věnuje problematice automatizace testování projektu napsaného v architektuře mikroslužeb s dodaným uživatelským webovým rozhraním.
V rámci testování budeme uvažovat pouze následující kategorie:

\begin{dl}
   \item[Statické testování kódu] – testování kvality kódu, nevyžaduje runtime. \TODO{citovat},
   \item[Jednotkové testování] – testování nejmenších částí implementace (funkce, třídy aj.),
   \item[Integrační testování] – testování integrací v rámci jedné \g{MS} i mezi \g{MS},
   \item[Funkční testování] – testování funkčnosti bez znalosti interní implementace~\cite{testtypes} dle testovacích scénářů,
   \item[Testování spolehlivosti] – testování způsobu chování v případech přetížení/výpadku a schopnost se obnovit po výpadku~\cite{testtypes2}.
\end{dl}

V ideálním případě je snaha mít každý automatizovaný typ testování automaticky spouštěný ve správnou chvíli během vývojového cyklu a nasazení sytému.
V případě neexistence možnosti vytvořit automatizaci nebo automacké spouštění, bude potřebný typ testování
Akceptovaným bude rovněž dokumentace s


Existují i další typy testů, jež se mohou automatizovat – testy výkonu, bezpečnosti, kompatibility, přenositelnosti apod.
V dané práci se jimi zabývat nebude.


\section{Nasazování}\label{sec:msa-deployment}

\subsection{Správa zdrojového kódu}\label{subsec:msa-deployment-code}
- Udržování zdrojových souborů jako monorepo / multirepo.

\section{Monitorování}\label{sec:msa-monitoring}

Monitorování je důležitou součástí vývoje a přizpůsobování mikroslužeb a aplikací obecně.
Dává přehled o stavu sytému a může napomáhat predikovat například budoucí zátěž na základě předchozích dat.
Sledované hodnoty můžeme rozdělit na tři hlavní skupiny~\cite{msactions}:

\begin{dl}
   \item [Metriky] – měřitelné hodnoty latence, chyb, zátěže a saturace systému.
   Může se jednat například o počet požadavků na server, počet chyb, objemu předaných dat, pokusů o opakované zpracování informací, využitou paměť a další hodnoty~\cite{msactions}.
   Takové historické informace se následně mohou podílet na přizpůsobování sytému dle nároků a potřeb.
   \item [Logy] – informace o uskutečněných událostech.
   \item [Stopy] – detailní záznamy o chybách a jejich původu.
\end{dl}

Na takovém rozdělení se zakládá například i populární~\cite{grafanapop} platforma Grafana~\cite{grafana}.

Monitorování v architektuře mikroslužeb je samostatná oblast s mnoha způsoby realizace, anazýzy a zpracování.
Vzhledem k širokému rozsahu nebude blíže zkoumána, až na návrhový vzor \h{Health Check}.


\subsection{Health check}\label{subsec:msa-monitoring-healthcheck}

Vzor \h{Health Check} je jednoduchým nástrojem pro monitorování stavu mikroslužby~\cite{healthcheck}.
Jedná se o speciální \g{API} GET rozhraní, které poskytuje základní informace o stavu mikroslužby.
Může se jednat například o název, verzi, stavu komunikce, stavu fyzického zařízení a další informace~\cite{healthcheck}.

Takové rozhraní, i když poskytuje jednoduchou kotrolu, tak nese i nevýhodu v podobě nedostupnosti, pokud celá mikroslužba nefunguje.

\section{Klientská část}\label{sec:testing-client-app}

Na straně klinsta se bude provádět testování

\subsection{Jednotkové testy}\label{subsec:testing-client-app-unit}

Obdobně, jako v případě každé \g{MS} pro jednotlivé části implementace byly napsány jednotkové testy, které se prováděly buď manuálně, nebo automaticky (vynuceně) během každé \enquote{push} akce v \h{git}.
Vynucená kontrola byla řešena s pomocí \h{git} hooks.

Následně byla automatizována kontrola základních interakcí a průchodů \g{MS} pomocí \TODO{Vybrat nástroj} (viz podkapitola Funkční testování).
Ná závěr byla provedeno akceptační testování (viz podkapitola Uživatelské akceptační testování).



\subsection{Funkční testování}\label{subsec:testing-client-app-functional}


\subsection{Uživatelské akceptační testování}\label{subsec:testing-client-app-acceptance}
Poslední fází je \g{UAT}, neboli uživatelské akceptační testování.
Daný typ testování se stěží nahrazuje automatickou formou, protože jde zejména o ověřování splněnosti požadavků klientem.

