

Softwarová architektura jako pojem není přesně definována \cite{softarch}

Arichitektura může být popsána jako jistý řád a pravidla, dle kterých je celá aplikace postavena.
Existují nějaké šablony architektury, které jsou vhodné pro vývoj aplikací


V dané kapitole bude popisována především \gls{MSA} a bude porovnávana s jinými architekturami, které by ji potenciálně mohly nahradit.
Tyto vybrané architektury – \gls{MA}, \gls{SOA}\footnote{též architektura orientovaná na služby}, \gls{SA} – budou zkoumány vzhledem k přístupu k určitým aspektům, poskytovaným možnostem a výhodám a nevýhodám vůči \gls{MSA}.



Služba (bez vazby na architekturu) je chápána jako atomicky fungující celek, z větší části nezávislý na ostatních – soutředí se na konkrtétní funkcionalitu, má vlastní databázi (nebo schéma) s výhradným přístupem a veškerá komunikace probíhá přes striktně definované rozhraní.
Jakýkoliv jiný vliv, než přes dohodnuté rozhraní, musí být eliminován.

Jelikož existuje nespočetné množství modifikací výše uvedených architektur, bude se předpokládat, že se jedná o takto definované instance:


%https://www.ibm.com/cloud/learn/soa#toc-soa-vs-mic-BjTfju28

\begin{dl}
   \item[\gls{MA}] – jednoprocesový program atomické povahy – nelze z něho jednoduše vyčlenit funkční celky, které by se daly beze změn využívat v jiných programech.
   Obsahuje globální jednorázové připojení k databázi, které se provádí během startu programu a kontroluje aktuální stav migrací (případně vykonává chybějící).
   \item[\gls{SOA}] – jednoprocesový program s interním rozdělením založeným na \gls{ESB} – sběrnicí určenou pro centralizaci komunikace mezi službami.
   Jednotlivé služby tvoří samostatně fungující moduly, jež veškerou komunikaci provádí skrz přesně definované rozhraní \gls{ESB}.
   Obdobně, jako u \gls{MA} existuje jedno
   \item[\gls{MSA}] –
   \item[\gls{SA}] – architektura aplikace, která postrádá neustále běžící serverový proces a je pouze rozmístěna na \gls{FaaS} řešení.
   Jinými slovy funkcionalita není spuštěna v nepřerušovaném prostředí (jako démon), ale je dostupná na požádání.
   Při dotazu (například REST) je vytvořena potřebná instance aplikace, případně navázáno databázové spojení, vykonán požadavek a následně je instance odstraněna.
   Taková aplikace nemůže být stavová v obvyklém slova smyslu.
   Kvůli průběhu zpracování se rovněž nehodí pro náročné aplikace. \TODO{zdroj}
\end{dl}

Pro porovnání architektur bylo vybráno několik klíčových pojmů, které mohou během návrhem a implementací programu mít nějvětší vliv na rozhodování o výběru architektury.

+- MA OR 29


\begin{dl}
   \item[Datové úložiště] – využití persistentního úložiště (SQL/NoSQL databáze) pro zápis a čtení informací.
   Počáteční inicializace databáze, vytvoření struktury, migrace
\end{dl}
\begin{ul}
   \item \gls{MA} – není náročná pro testování.
   \item \gls{SOA} – rovněž může být testována
   \item \gls{MSA} – plnohodnotné testování celé struktury vyžaduje její nasazení na server.
\end{ul}

\begin{dl}
   \item[Nezávislost] – popis
\end{dl}
\begin{ul}
   \item \gls{MA} – něco umí
   \item \gls{SOA} – něco umí
   \item \gls{MSA} – něco umí
\end{ul}

\begin{dl}
   \item[Radikální změny] – složitost změny nebo přidání business logiky do aplikace, která by měl dopad na větší část dosavadní aplikace.
\end{dl}
\begin{ul}
   \item \gls{MA} –
   Z hlediska datového úložiště bude potřeba dělat pouze jednu migraci
   \item \gls{SOA} – něco umí
   \item \gls{MSA} – něco umí
\end{ul}

\begin{dl}
   \item[Tolerace chyb] – popis
\end{dl}
\begin{ul}
   \item \gls{MA} – něco umí
   \item \gls{SOA} – něco umí
   \item \gls{MSA} – něco umí
\end{ul}

\begin{dl}
   \item[Komunikační latence] – popis
\end{dl}
\begin{ul}
   \item \gls{MA} – něco umí
   \item \gls{SOA} – něco umí
   \item \gls{MSA} – něco umí
\end{ul}

\begin{dl}
   \item[Škálování] – popis
\end{dl}
\begin{ul}
   \item \gls{MA} – něco umí
   \item \gls{SOA} – něco umí
   \item \gls{MSA} – něco umí
\end{ul}

\begin{dl}
   \item[Konzistence] – popis
\end{dl}
\begin{ul}
   \item \gls{MA} – něco umí
   \item \gls{SOA} – něco umí
   \item \gls{MSA} – něco umí
\end{ul}

\begin{dl}
   \item[Nasazení] – popis
\end{dl}
\begin{ul}
   \item \gls{MA} – něco umí
   \item \gls{SOA} – něco umí
   \item \gls{MSA} – něco umí
\end{ul}

\begin{dl}
   \item[Testování] – schopnost psaní automatizovaných testů, které se vyplní při spuštění aplikace.
\end{dl}
\begin{ul}
   \item \gls{MA} – něco umí
   \item \gls{SOA} – něco umí
   \item \gls{MSA} – něco umí
\end{ul}
