

Softwarová architektura jako pojem není exaktně definována, každý vývojář k ní může přistupovat jinak  \cite{softarch}


Arichitektura může být popsána jako jistý řád a pravidla, dle kterých je celá aplikace postavena.
Existují nějaké šablony architektury, které jsou vhodné pro vývoj aplikací


Ať se architektura chápe jakkoliv, obecně se dá říct, že její stanovení a dodržování

V počátečních fázích projektu se dá
Architektura

https://martinfowler.com/bliki/DesignStaminaHypothesis.html


V dané kapitole bude popisována především \g{MSA} a bude porovnávana s jinými architekturami, které by ji potenciálně mohly nahradit.
Tyto vybrané architektury – \g{MA}, \g{SOA}\footnote{též architektura orientovaná na služby}, \g{SA} – budou zkoumány vzhledem k přístupu k určitým aspektům, poskytovaným možnostem a výhodám a nevýhodám vůči \g{MSA}.

Vzhledem k dříve použitému jazyku TypeScript/JavaScript budou i mikroservisy zkoumány se zaměřením na tento jazyk.


Všechny 3 architektury mají jeden společný pojem – služba.
Služba (bez vazby na konkrétní architekturu) v této práci je chápána jako atomicky fungující celek, z větší části nezávislý na ostatních – soutředí se na konkrtétní funkcionalitu, má vlastní databázi (nebo schéma) s výhradným přístupem a veškerá komunikace probíhá přes striktně definované rozhraní.
Jakýkoliv jiný vliv, než přes dohodnuté rozhraní, musí být eliminován.

\begin{figure}[htbp]
   \centering
   \includegraphics[max width=\textwidth]{assets/draft-service.png}
   \caption{Abstraktní znázornění služby}\label{pic:service-abstract}
\end{figure}

Jelikož existuje nespočetné množství modifikací výše uvedených architektur, bude se předpokládat, že se jedná o takto definované instance:


%https://www.ibm.com/cloud/learn/soa#toc-soa-vs-mic-BjTfju28

\begin{dl}
   \item[\g{MA}] – jednoprocesový program atomické povahy – nelze z něho jednoduše vyčlenit funkční celky, které by se daly beze změn využívat v jiných programech.
   Obsahuje globální jednorázové připojení k databázi, které se provádí během startu programu a kontroluje aktuální stav migrací (případně vykonává chybějící).
   \item[\g{SOA}] – jednoprocesový program s interním rozdělením založeným na \g{ESB} – sběrnicí určenou pro centralizaci komunikace mezi službami.
   Jednotlivé služby tvoří samostatně fungující moduly, jež veškerou komunikaci provádí skrz přesně definované rozhraní \g{ESB}.
   Obdobně, jako u \g{MA} existuje jedno
   \item[\g{MSA}] –
   \item[\g{SA}] – architektura aplikace, která postrádá neustále běžící serverový proces a je pouze rozmístěna na \g{FaaS} řešení.
   Jinými slovy funkcionalita není spuštěna v nepřerušovaném prostředí (jako démon), ale je dostupná na požádání.
   Při dotazu (například REST) je vytvořena potřebná instance aplikace, případně navázáno databázové spojení, vykonán požadavek a následně je instance odstraněna.
   Taková aplikace nemůže být stavová v obvyklém slova smyslu.
   Kvůli průběhu zpracování se rovněž nehodí pro náročné aplikace. \TODO{zdroj}
\end{dl}

Pro porovnání architektur bylo vybráno několik klíčových pojmů, které mohou během návrhem a implementací programu mít nějvětší vliv na rozhodování o výběru architektury.

+- MA OR 29


\begin{dl}
   \item[Datové úložiště] – využití persistentního úložiště (SQL/NoSQL databáze) pro zápis a čtení informací.
   Počáteční inicializace databáze, vytvoření struktury, migrace
\end{dl}
\begin{ul}
   \item \g{MA} – není náročná pro testování.
   \item \g{SOA} – rovněž může být testována
   \item \g{MSA} – plnohodnotné testování celé struktury vyžaduje její nasazení na server.
\end{ul}

\begin{dl}
   \item[Nezávislost] – popis
\end{dl}
\begin{ul}
   \item \g{MA} – něco umí
   \item \g{SOA} – něco umí
   \item \g{MSA} – něco umí
\end{ul}

\begin{dl}
   \item[Radikální změny] – složitost změny nebo přidání business logiky do aplikace, která by měl dopad na větší část dosavadní aplikace.
\end{dl}
\begin{ul}
   \item \g{MA} –
   Z hlediska datového úložiště bude potřeba dělat pouze jednu migraci
   \item \g{SOA} – něco umí
   \item \g{MSA} – něco umí
\end{ul}

\begin{dl}
   \item[Tolerace chyb] – popis
\end{dl}
\begin{ul}
   \item \g{MA} – něco umí
   \item \g{SOA} – něco umí
   \item \g{MSA} – něco umí
\end{ul}

\begin{dl}
   \item[Komunikační latence] – popis
\end{dl}
\begin{ul}
   \item \g{MA} – něco umí
   \item \g{SOA} – něco umí
   \item \g{MSA} – něco umí
\end{ul}

\begin{dl}
   \item[Škálování] – popis
\end{dl}
\begin{ul}
   \item \g{MA} – něco umí
   \item \g{SOA} – něco umí
   \item \g{MSA} – něco umí
\end{ul}

\begin{dl}
   \item[Konzistence] – popis
\end{dl}
\begin{ul}
   \item \g{MA} – něco umí
   \item \g{SOA} – něco umí
   \item \g{MSA} – něco umí
\end{ul}

\begin{dl}
   \item[Nasazení] – popis
\end{dl}
\begin{ul}
   \item \g{MA} – něco umí
   \item \g{SOA} – něco umí
   \item \g{MSA} – něco umí
\end{ul}

\begin{dl}
   \item[Testování] – schopnost psaní automatizovaných testů, které se vyplní při spuštění aplikace.
\end{dl}
\begin{ul}
   \item \g{MA} – něco umí
   \item \g{SOA} – něco umí
   \item \g{MSA} – něco umí
\end{ul}
