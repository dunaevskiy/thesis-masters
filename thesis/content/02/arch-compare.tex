Pro porovnání architektur bylo vybráno několik klíčových pojmů, které mohou během návrhem a implementací programu mít nějvětší vliv na rozhodování o výběru architektury.

+- MA OR 29


\begin{dl}
   \item[Datové úložiště] – využití persistentního úložiště (SQL/NoSQL databáze) pro zápis a čtení informací.
   Počáteční inicializace databáze, vytvoření struktury, migrace
\end{dl}
\begin{ul}
   \item \g{MA} – není náročná pro testování.
   \item \g{SOA} – rovněž může být testována
   \item \g{MSA} – plnohodnotné testování celé struktury vyžaduje její nasazení na server.
\end{ul}

\begin{dl}
   \item[Nezávislost] – popis
\end{dl}
\begin{ul}
   \item \g{MA} – něco umí
   \item \g{SOA} – něco umí
   \item \g{MSA} – něco umí
\end{ul}

\begin{dl}
   \item[Radikální změny] – složitost změny nebo přidání business logiky do aplikace, která by měl dopad na větší část dosavadní aplikace.
\end{dl}
\begin{ul}
   \item \g{MA} –
   Z hlediska datového úložiště bude potřeba dělat pouze jednu migraci
   \item \g{SOA} – něco umí
   \item \g{MSA} – něco umí
\end{ul}

\begin{dl}
   \item[Tolerace chyb] – popis
\end{dl}
\begin{ul}
   \item \g{MA} – něco umí
   \item \g{SOA} – něco umí
   \item \g{MSA} – něco umí
\end{ul}

\begin{dl}
   \item[Komunikační latence] – popis
\end{dl}
\begin{ul}
   \item \g{MA} – něco umí
   \item \g{SOA} – něco umí
   \item \g{MSA} – něco umí
\end{ul}

\begin{dl}
   \item[Škálování] – popis
\end{dl}
\begin{ul}
   \item \g{MA} – něco umí
   \item \g{SOA} – něco umí
   \item \g{MSA} – něco umí
\end{ul}

\begin{dl}
   \item[Konzistence] – popis
\end{dl}
\begin{ul}
   \item \g{MA} – něco umí
   \item \g{SOA} – něco umí
   \item \g{MSA} – něco umí
\end{ul}

\begin{dl}
   \item[Nasazení] – popis
\end{dl}
\begin{ul}
   \item \g{MA} – něco umí
   \item \g{SOA} – něco umí
   \item \g{MSA} – něco umí
\end{ul}

\begin{dl}
   \item[Testování] – schopnost psaní automatizovaných testů, které se vyplní při spuštění aplikace.
\end{dl}
\begin{ul}
   \item \g{MA} – něco umí
   \item \g{SOA} – něco umí
   \item \g{MSA} – něco umí
\end{ul}
