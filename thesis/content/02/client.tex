% READY, REWRITE

\section{Mikroslužby webové aplikace}\label{sec:msa-client}

Posledním tématem věnovaným obecné analýze \g{MS} je otázka mikroslužeb klientských webových aplikací.
Ve webovém prostředí se většinou jedná o běžnou \h{client-server} komunikaci – existuje zdrojový server a koncový klient, který zobrazuje a zpracovává stahovaná data.
Omezíme se na skupinu nejtypičějších souborů – \h{.html} \h{.css} \h{.js} a binární data (obrázky, písmo, zvuk, videa).
Obsah pro uživatele v takovém případě může být dělen následovně:

\begin{dl}
   \item[Statický obsah] – statické soubory, které jsou stahovány klientem přes HTTP protokol během základního načtítání stránky\footnote{Základním načtením stránky je zde myšleno načtení dat bez interpretace JavaScript souborů, které mění stránku dodatečně (a obdobných technologií)}.
   \item[Dynamicky generovaný obsah na straně serveru] – například HTML šablony s možnými datovými sety (viz \g{SSG}).
   \item[Dynamicky generovaný obsah na straně klienta] – například \g{AJAX} načítání obsahu po prvotním načtení stránky.
\end{dl}

V prvních dvou případech dělení na dříve definované služby (atomický celek s komunikací pouze přes strikně definované rozhraní, jenž může být vyvíjen nezávisle na ostatních částech aplikace) není úplně vhodné – všechno je ve výsledku opět řízeno serverem (logika sestavování šablon).
V posledním případě však můžeme vytvořit celky, jež mohou vykazovat samostatné, izolované chování, a jejich životní cyklus by byl plně řízen klientem.

Bohužel na rozdíl od serverových aplikací, kde pro každou část je dostupný vlastní proces, na klienstké straně existuje pouze jeden (pokud budeme brát v úvahu nejrozšiřenější prohlížeč Chrome~\cite{browserstats}), který zpracovává všechny požadavky~\cite{chromiumprocesses}.
Separovat takové části úplně nelze - musíme počítat se společným prostorem pro vykonávání JavaScript souborů a společné \g{CSS} styly.

V rámci klientské aplikaci proto musíme předefinovat službu jako samostatný celek, komunikující pouze přes přesně dané rozhraní, vlastnící vlastní část datového úložiště, ale sdílící společné prostředky.

Pokud pomineme existenci \h{<iframe>} tagu v HTML, který do jisté míry vyhovuje, ale přináší i jistá omezení~\cite{iframes}, tak se nabízí možnost definovat menší logické balíčky, stahované přes \g{AJAX} dle vyžádání po načtení stránky.
Vývoj daných balíčků by byl řízen určitými pravidly, při dodržení kterých by mohly být opakovaně používány na stránce nebo dokonce ve více aplikacích.

I když existence takových celků je možná, tak stále chybí jejich inicializace na stránce.
Z tohoto důvodu se nelze vyhnout implementaci obecného monolitu, jenž by se načítal jako inicializátor obsahu.
Hlavním úkolem by pro něj bylo řízení načítání a interpretace balíčků dle určitých příznaků (například URL adresa, nebo vlastní směrovací systém).
V případě potřeby, by měl vytvářet a poskytovat komunikační kanály mezi samostatnými celky.

Ve výsledku se balíčky velice podobají chováním všemožných pluginů, jen se striktním omezením z hlediska integrace s prostředím.

Na základě obdobných úvah existuje manuál Micro Frontends, který realizuje myšlenku s pomocí základních JavaScript možností, nabývá však i svých omezení a nevýhod~\cite{microfrontends}.
