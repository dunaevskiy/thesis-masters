\section{Komunikace \g{MS}}\label{sec:msa-communication}

Mikroslužby jsou vždy nezávislé nezávislé z hlediska procesů \TODO{cite}, ale pro  musí udržovat kompatibilní \g{API} vůči jejich prostředí.
Takové závislosti se nejlépe kontrolují verzováním jednotlivých mikroservis s pomocí sématického verzování.
Zde je třeba najít konkrétní nástroj pro tuto kontrolu.

Komunikace dle typu:

\begin{dl}
   \item [Synchronní/Asynchronní]
   \item [1:1/1:n] –
\end{dl}





\subsection{Komunikace}

Na nejvyšší úrovni abstrakce komunikaci mikroslužeb je možné zorganizovat dvěma hlavními způsoby~\cite{choreovsorch}:

\begin{ul}
   \item Orchestrace
   \item Choreografie
\end{ul}




- sync/async
- latence
- problém řetězových závislotí



\section{Udržování závislostí}\label{sec:msa-dependencies}
