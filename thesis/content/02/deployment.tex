\section{Nasazování}\label{sec:msa-deployment}

Nasazení mikroslužeb je jedna ze závěrečných fází vývoje systému, pokrývá široké spektrum možností – rozmístění produkčních balíčků na samostatných fyzických serverech, cloud-řešení, v klasterech, s vyvažováním zátěže, replikací apod.
Daná podkapitola se bude věnovat pouze společným rysům – přípravě pro všechny typy nasazení – a kontejnerizací, jenž je vyžadována zadáním diplomové práce.


\subsection{Správa zdrojového kódu}\label{subsec:msa-deployment-code}

Služba, případně mikroslužba, z hlediska chování byla v této práci definována jako samostatný, atomicky fungující celek.
Toto se týká i existence spuštěné instance – musí být schopna existovat samostatně (alespoň z hlediska propojení s ostatními službami).
Taková vlastnost má důležitý dopad na potřebné horizontální škálování a rozložení zátěže v nejvyužívanějších částech systému~\cite{monomulti}.

I přes takové předpoklady samotnou správu zdrojového kódu je možné uspořádat jak do jednoho (monorepozitář) tak i více repozitářů (za předpokladu, že se využívá \g{VCS} git).
Oba přístupy mají své výhody a nevýhody.


\begin{dl}
   \item[Monorepozitář] – existence jednoho repozitáře, kde jednotlivé služby jsou umístěny do vlastních složek.

   Výhody
   \begin{ul}
      \item Celý produkt je uchováván na jednom místě - pro tým to může znamenat lepší podmínky pro testování a vývoj, jelikož mají dostupné všechny vyvíjené častí všemi týmy~\cite{monomulti}.
      \item V případě refaktorování nebo testování systému je možné vše provádět na jednom místě~\cite{monomulti}.
      \item Vývojové prostředí je možné nastavit tak, že všechny služby budou sdílet stejnou konfiguraci (kontrola kvality kódu, formátování a další)~\cite{monomulti}.
   \end{ul}

   Nevýhody
   \begin{ul}
      \item Obrovské množeství služeb v jednom repozitáři může znepřehlednit vývoj – git historie se týká nejen jednoho produktu, ale všech – tagy a větve budou sdílené pro všechny služby.
      \item Existence možného zásahu do zdrojovýho kódu služeb vývojáři, jež pro to nemají formální oprávnění~\cite{monomulti}.
   \end{ul}

   \item[Více repozitářů] – každá služba má vlastní repozitář, který je naprosto soběstačný.

   Výhody
   \begin{ul}
      \item Přesné rozdělení vývojových částí a přístupů mezi jednotlivými vývojovými týmy~\cite{monomulti}.
      \item Přehledná git historie pro každou službu včetně tagů.

   \end{ul}

   Nevýhody
   \begin{ul}
      \item Absence jednoduchého spuštění všech částí projektu~\cite{monomulti}.
      \item V případě existence sdílených zdrojových kódů, je nutné tuto situaci řešit jinými způsoby, než extraci do sdíleného prostoru nadsložek.
   \end{ul}
\end{dl}


Ze subjektivní zkušenosti je do tohoto přehledu možné přidat ještě jeden způsob vedení projektu.
Spočívá v kombinaci obou přístupů s využitím \h{git submodules}.
V takovém případě struktura projektu je oraganizována následovně:

\begin{ul}
   \item Každá služba má svůj vlastní repozitář a jejich vývoj je nezávislý.
   \item Existuje jeden repozitář, který importuje všechny služby jako git submoduly a přidává vhodnou konfiguraci vně submodulů (například \h{docker-compose}, testy apod.).
\end{ul}

Takový přístup přináší některé výhody monorepozitáře, ale přístup k jednotlivým službám je samostatný a může být omezován právy pro každý repozitář.
Tzn.
v případě snahy o jednotnou statickou analýzu kódu (či obdobné záležitosti) je možné přesunou konfigurace ze všech repozitářů mikroslužeb do společného repozitáře se submoduly a odkázat původní konfigurace a konfiguraci v nadsložce.
Tímto zásahem se ale znemožní separátní vývoj a bude potřeba vždy stahovat hlavní repozitář a z toho submodul, nad kterým se má provádět vývoj.



\subsection{Kontejnerizace}\label{subsec:msa-deployment-containerization}
Kontejnerizace je jistý způsob virtualizace za použitím menšího množství systémových zdrojů, než u plnohodnotné virtualizace~\cite{kontejnerizace}.
Základní myšlenka spočívá v přípavě soběstačného balíčku (obrazu) s programem, který by se dal spouštět jednoduchým vytvářením nové instance (kontejneru) v jakémkoliv prostředí, které poskytuje dostatečné základní rozhraní~\cite{dockercontainer}.

V případě JavaScript aplikací založených na Node.js prostředí a využitím služby Docker se bude jednat o následující proces:

\begin{dl}
   \item [Stažení výchozího prostředí] – obraz prostředí, ze kterého bude mikroslužba vycházet, v tomto případě \h{node}.
   \item [Přidání npm/yarn závislostí] – do obrazu jsou staženy všechny vnější závislosti.
   \item [Přidání a sestavení samotného projektu] - do obrazu jsou přidány zdrojové soubory samotného programu a spuštěno sestavení.
   \item [Otevření potřebných komunikačních kanálů] - mikroslužba komunikuje na vlastním, předem určeném portu.
   Po vytvoření instance tento port musí být otevřen pro komunikaci s kontejnerem.
   \item [Definování příkazu pro spuštění] – definice sekvence příkazů, které během vytvoření kontejneru připraví a spuští příkaz pro start webového serveru.
\end{dl}

Takto se musí připravit každá služba v rámci systému.
Spuštění celého projektu následně může být řízeno \h{docker-compose}, který kromě potřebých mikroslužeb ještě poskytne databáze a další potřebné programy třetích strav.
