\section{Nasazování}\label{sec:msa-deployment}

Nasazení mikroslužeb je jedna ze závěrečných fází vývoje systému, pokrývá široké spektrum možností – rozmístění produkčních balíčků na samostatných fyzických serverech, cloud-řešení, v klasterech, s vyvažováním zátěže, replikací apod.
Daná podkapitola se bude věnovat pouze společným rysům – přípravě pro všechny typy nasazení – a kontejnerizací, jenž je vyžadována zadáním diplomové práce.


\subsection{Správa zdrojového kódu}\label{subsec:msa-deployment-code}

Služba, případně mikroslužba, z hlediska chování byla v této práci definována jako samostatný, atomicky fungující celek.
Toto se týká i existence spuštěné instance – musí být schopna existovat samostatně (alespoň z hlediska propojení s ostatními službami).
Taková vlastnost má důležitý dopad na potřebné horizontální škálování a rozložení zátěže v nejvyužívanějších částech systému~\cite{monomulti}.

I přes takové předpoklady samotnou správu zdrojového kódu je možné uspořádat jak do jednoho (monorepozitář) tak i více repozitářů (za předpokladu, že se využívá \g{VCS}).
Oba přístupy mají své výhody a nevýhody.


\begin{dl}
   \item[Monorepozitář] – existence jednoho repozitáře, kde jednotlivé služby jsou umístěny do vlastních složek.

   Výhody
   \begin{ul}
      \item
   \end{ul}

   Nevýhody
   \begin{ul}
      \item
   \end{ul}

   \item[Více repozitářů] – každá služba má vlastní repozitář, který je naprosto soběstačný.

   Výhody
   \begin{ul}
      \item
   \end{ul}

   Nevýhody
   \begin{ul}
      \item
   \end{ul}
\end{dl}




\subsection{Kontejnerizace}\label{subsec:msa-deployment-containerization}
Kontejnerizace je jistý způsob virtualizace za použitím menšího množství systémových zdrojů, než u plnohodnotné virtualizace~\cite{kontejnerizace}.


At už se jedná o jakýkoliv způsob nasazování mikroservistní architektury, tak je důležité poskytovat i vhodně detailní dokumentaci, zejména diagramy komunikace závislostí mikroslužeb.
