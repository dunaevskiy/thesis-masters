\section{Využití \gls{MSA} pro modelování světa}

Pro jedince nejspíš neexistuje nic přirozenějšího, než prostředí, ve kterém se pohybuje a kterému rozumí - od osobních věcí a proseců, jež musí vykonávat, až po uspořádání světa – město, stát, planeta, politické a sociální vztahy, komunikace s institucemi a doprovod dětí do školy.

Všechny tyto činnosti můžeme popsat pomocí subjektů – účastníky procesů, rozhraní, které poskytují pro komunikaci, a bezprostředně zpráv, které se předávají v rámci komunikace.

Každý subjekt je ideálně autonomní -

Daný konecept naprosto přesně napodobuje \gls{MSA}.
Jednotlivé subjekty jsou mikroservisy, mají vlastní rozhraní a vysílají informace v přesně definovaných (i když občas dynamických) strukturách.
Zároveň mohou zachytit a zpracovat pouze takové zprávy, které byly určeny pro jejich použití a formátu kterých je popsaný v jejich vnitřní logice.

Oproti tomu \gls{SOA} tento koncept zjednodušuje.
Subjekty již nejsou samostatně potenciálně fungující celky, nýbrž moduly jednoho většího.
Obdobně se mohou přidávat a odebírat, ale jejich komunikačnní rozhraní je pravděpodobně jednodušší.
...

Monolitní architektura představuje mnohem větší zjednodušení.
V rámci modelování světa se dá v jistém smyslu přirovnat k Laplaceovu démonu – existující systém je samostatný, dále nedělitelný, má vědomosti a informaci o všem, co se děje, .
V praxi je taková struktura ovlivněna vnějšími vstupy, ale interní komunikace a potřeba znalostí o všech procesech zůstává.



Na základě výše uvedených informací by logicky bylo nejjednodušší vždy vytvářet pouze \gls{MSA} architekturu, protože je pro pochopení nejsnažší kvůli zkušenostem z každodenního života.
Každý subjekt má svoje potřeby a o zbylé aspekty se nestará.
Problém nastává kvůli komplexnosti takového konceptu.
Jakékoliv rozdělení celku implicitně předpokládá nové náklady pro definování komunikace.
Pokud se dobře nezvolí hloubka rozdělení, tak se definuje mnoho zbytečné komunikace, která přinese pouze náklady navíc.

