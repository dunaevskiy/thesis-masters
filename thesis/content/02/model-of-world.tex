\section{Využití \g{MSA} pro modelování světa}\label{sec:msa-model-of-world}

Pro člověka nejspíš neexistuje nic přirozenějšího, než prostředí, ve kterém se pohybuje a kterému rozumí – od osobních věcí a proseců, jež musí vykonávat, až po uspořádání světa – město, stát, planeta, politické a sociální vztahy, komunikace s institucemi a interakce s rodinou.

Všechny tyto činnosti můžeme popsat pomocí subjektů – samostatných účastníků procesů, rozhraní, které poskytují pro komunikaci, a zpráv\footnote{Zprávou se rozumí informace poskytnutá v samostatné struktuře – například \g{JSON} nebo \g{XML}}, jež se předávají v rámci komunikace.
Každý subjekt je skupinou izolovaných funkcí s různě kompikovanou sadou komunikačních kanálů.
Může mít svoje potřeby a může vytvářet podněty pro ostatní subjekty.
Ne každý subjekt je schopen zpracovávat veškerou informaci, která k němu přichází od jiných subjektů.

Daný konecept naprosto přesně napodobuje \g{MSA}.
Jednotlivé subjekty jsou mikroslužby, mají vlastní rozhraní a vysílají informace v přesně definovaných strukturách.
Mohou existovat samostatně, i když jejich smysl existence nemusí existovat.
Zároveň jsou schopny zachytit a zpracovat zprávy, které byly určeny pro jejich použití a formát kterých je popsaný ve vnitřní logice.

V případě mírného zjednodušení se můžeme dostat ke konceptu \g{SOA}.
Subjekty zachovávají způsob komunikace s pomocí zpráv, ale již nejsou samostatní, potenciálně fungující celky, nýbrž moduly jednoho většího bloku.
Komunikace u takové architektury může být centrállně řízena \g{ESB}.\cite{soavsmsa}

Monolitní architektura v provonání se \g{SOA} zjednodušuje i samotné rozdělení do modulů.
Stále se jedná o samostatně fungující celek, ale vnitří struktura už postrádá moduly s odděleným rozhraním komunikace s využitím zpráv.
V rámci modelování světa se dá představit jako interně nedělitelný celek, který pouze poskytuje rozhraní pro komunikaci, tudíž je to subjekt a v rámci \g{MSA} může představovat mikroslužbu.

Na základě výše uvedených informací by logicky bylo nejjednodušší vždy vytvářet pouze \g{MSA} architekturu, protože je pro pochopení nejsnažší kvůli zkušenostem z každodenního života.
Každý subjekt má svoje potřeby (potřeva vytvořit zprávu, potřeba zpracovat příchozí právu) a o zbylé aspekty se nestará.
Problém nastává kvůli komplexnosti takového konceptu.
Jakékoliv rozdělení celku implicitně předpokládá nové náklady pro definování komunikace, které jsou náročné pro představu.
Proto může být nejvýhodnější začít architektorou s nejvíc redukovaným rozdělením – monolitem a dle potřeby měnit hloubku rozdělení.

Navíc při detailním zkoumání ve všech výše uvedených architekturách se dá vypozorovat rekurzivita.
Monolitická aplikace nebo \g{SOA} aplikace může tvořit mikroslužbu v rámci \g{MSA}.
\g{MSA} aplikace může tvořit mikroslužbu jiné \g{MSA} a fungovat jako monolit pro ostatní.
