\section{Monitorování}\label{sec:msa-monitoring}

Monitorování je důležitou součástí vývoje a přizpůsobování mikroslužeb a aplikací obecně.
Dává přehled o stavu sytému a může napomáhat predikovat například budoucí zátěž na základě předchozích dat.
Sledované hodnoty můžeme rozdělit na tři hlavní skupiny~\cite{msactions}:

\begin{dl}
   \item [Metriky] – měřitelné hodnoty latence, chyb, zátěže a saturace systému.
   Může se jednat například o počet požadavků na server, počet chyb, objemu předaných dat, pokusů o opakované zpracování informací, využitou paměť a další hodnoty~\cite{msactions}.
   Takové historické informace se následně mohou podílet na přizpůsobování sytému dle nároků a potřeb.
   \item [Logy] – informace o uskutečněných událostech.
   \item [Stopy] – detailní záznamy o chybách a jejich původu.
\end{dl}

Na takovém rozdělení se zakládá například i populární~\cite{grafanapop} platforma Grafana~\cite{grafana}.

Monitorování v architektuře mikroslužeb je samostatná oblast s mnoha způsoby realizace, anazýzy a zpracování.
Vzhledem k širokému rozsahu nebude blíže zkoumána, až na návrhový vzor \h{Health Check}.


\subsection{Health check}\label{subsec:msa-monitoring-healthcheck}

Vzor \h{Health Check} je jednoduchým nástrojem pro monitorování stavu mikroslužby~\cite{healthcheck}.
Jedná se o speciální \g{API} GET rozhraní, které poskytuje základní informace o stavu mikroslužby.
Může se jednat například o název, verzi, stavu komunikce, stavu fyzického zařízení a další informace~\cite{healthcheck}.

Takové rozhraní, i když poskytuje jednoduchou kotrolu, tak nese i nevýhodu v podobě nedostupnosti, pokud celá mikroslužba nefunguje.
