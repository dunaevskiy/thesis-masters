
\section{Testování a automatizace}

Daná kapitola se věnuje problematice automatizace testování projektu napsaného v architektuře mikroslužeb s dodaným uživatelským webovým rozhraním.
V rámci testování budeme uvažovat pouze následující kategorie:

\begin{dl}
   \item[Jednotkové testování] – testování nejmenších částí implementace (funkce, třídy aj.),
   \item[Integrační testování] – testování integrací v rámci jedné \g{MS} i mezi \g{MS},
   \item[Funkční testování] – testování funkčnosti bez znalosti interní implementace~\cite{testtypes} dle testovacích scénářů,
   \item[Testování výkonu] – testování systému v předem dané zátěži.
   \item[Testování spolehlivosti] – testování způsobu chování v případech přetížení/výpadku a schopnost se obnovit po výpadku~\cite{testtypes2}.
\end{dl}

V ideálním případě je snaha mít každý automatizovaný typ testování automaticky spouštěný ve správnou chvíli během vývojového cyklu a nasazení sytému.
V případě neexistence možnosti vytvořit automatizaci nebo automacké spouštění, bude potřebný typ testování
Akceptovaným bude rovněž dokumentace s


Existují i další typy testů, jež se mohou automatizovat – testy bezpečnosti, kompatibility, přenositelnosti apod.
V dané práci se jimi zabývat nebude.

