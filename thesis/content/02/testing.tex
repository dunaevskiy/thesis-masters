\section{Testování a automatizace}\label{sec:testing}

Problematika testování architektury mikroslužeb se v základu ničím neliší od testování jiných typů aplikací – existují určité typy, které je nutné vhodně aplikovat na poskytnuté zdroje.
Některé typy testů mohou plně podléhat automatizaci, něco bude vyžadovat dodání doplňujících materiálů a něco je lepší testovat manuálně kvůli lepším výsledkům testů.
V rámci testování \g{MSA} můžeme brát jako základ například následující kategorie:

\begin{dl}
   \item[Statické testování kódu] – nevyžaduje běh aplikace, jde například o testování kvality kódu, stylizace apod.~\cite{statictest}.
   V Node.js se o tuto část může starat \h{eslint} (pravidla psaní kódu) a \h{prettier} (formátování).
   \item[Jednotkové testování] – testování nejmenších částí implementace (funkce aj.)~\cite{unitinttest}.
   \item[Integrační testování] – testování integrací v systému~\cite{unitinttest}.
   V rámci mikroslužeb můžeme testovat integraci jak mezi nejmenšími jednotkami, tak i brát celou \g{MS} a~testovat její integraci se zbytkem aplikace.
   \item[Funkční testování] – testování funkčnosti bez znalosti interní implementace~\cite{testtypes}, například dle testovacích scénářů.
   Zde se může jednat třeba o otestování jednotlivých rozhraní v rámci spuštěné aplikace, nebo navazující sekvenci takových volání.
   \item[Testování spolehlivosti] – testování způsobu chování v případech přetížení/výpadku a schopnost se obnovit po výpadku~\cite{testtypes2}.
   \item[Testování zátěže] – testování výkonu aplikace s pomocí simulace aktivity systému~\cite{loadtest}.
\end{dl}

Existují i další typy testů, jež se mohou provádět nad \g{MSA}, vždy je nutné volit to, co je pro vyvíjený systém potřebné.
V ideálním případě je snaha mít každý automatizovaný typ testování automaticky spouštěný ve správnou chvíli během vývojového cyklu a~nasazení sytému.
