\chapter{Mikroservisní architektura - Správa dat}\label{ch:msa-data}

\chaptersummary{
   \begin{ul}
      \item způsoby integrace datových zdrojů do \g{MSA},
      \item tranzakční zpracování a agregace dat ve složitějších, neizolovaných strukturách,
      \item ...
   \end{ul}
}

V \g{MSA}...


\section{Databáze jako zdroj dat}\label{sec:msa-db-as-data-source}

Jako perzistentní úložiště pro \g{MSA} může sloužit databáze.
Pro návrh struktury ukládání dat se mohou používat dva návrhové vozry:
\begin{dl}
   \item [Databáze pro každou službu] – každý deklarovaná mikroslužba používá vlastní databázi (nebo schéma), do které má výhradní přistup.
   \item [Sdílená databáze] – běžná jednoduchá strktura, kde každá mikroslužba může měnit veškerá data.
\end{dl}

Takové přístupy se liší ve způsobu ukládání a přístupu k datům.




\section{Komplikace a řešení sdílené databáze}\label{sec:msa-db-issues}

Speciální oblast tvoří operace prováděné nad databází bez sdíleného přístupu pro mikroslužby.
Kromě logické separace datových zdrojů může docházet i k fyzickému rozdělení dat.
U databází na principu \g{ACID} může při některých pořadavcích docházet k porušení izolace a je třeba zajistit konzistenci.
Zároveň není možné pouze databázovými prostředky provádět agregaci dat mezi mikroslužbami.
\TODO{Popis}

\subsection{Agregace}\label{subsec:msa-db-aggregate}
% Jako jeden z možných pořadavků v rámci \g{MSA} je agregace mezi schématy, což v rámci oddělených služeb znamená netriviální úkol.
% Agregaci v tomto případě nemůžeme nechávat na databázi, protože může být fyzicky oddělena.

\subsection{Integritní omezení}\label{subsec:msa-db-integrity}
\subsection{Tranzakční zpracování - Saga}\label{subsec:msa-db-transaction}
Sagy, neboli příběhy, zajištují tranzakční zpracování napříč databázemi.
