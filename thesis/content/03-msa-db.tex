\chapter{Správa dat v architektuře mikroslužeb}\label{ch:msa-data}

\chaptersummary{
   \begin{ul}
      \item způsoby integrace datových zdrojů do \g{MSA},
      \item spojování dat napříč mikroslužbami,
      \item transakční zpracování napříč mikroslužbami.
   \end{ul}
}

Databáze jako datový zdroj je jedním z možných řešení pro persistentní ukládání dat s případným následným zpracováním.
Daná kapitola je věnována především dvěma návrhovým vzorům pro organizaci dat v relačních databázích dodržujících \g{ACID} vlastnosti a spolupráci mikroslužeb pro vyhodnocování pokročilých klientských požadavků.
Popisované databázové organizace dat se obecně nemusí týkat pouze architektury mikroslužeb, ale nachází zde svoje přímé uplatnění.



\section{Typy organizace zdrojů}\label{sec:msa-db-as-data-source}

V jednoduché monolitní architektuře využití relační databáze může být zcela přímočaré – jeden monolit se připojuje k jedné databázi, která spravuje veškerá potřebná data.
Mikroslužby však představují separaci na jednotlivé částí dle zvolené dekompozice a intuitivně přináší i myšlenku dělení původně jednoho datového zdroje na obdobně dekomponované celky.
Na základě této úvahy můžeme definovat dva návrhové vzory pro organizaci dat:

\begin{dl}
   \item [Sdílená databáze] – jedna instance databáze je plně přístupna pro všechny připojené mikroslužby~\cite{shareddb}.
   \item [Databáze pro každou službu] – každá deklarovaná mikroslužba používá vlastní databázi (nebo schémata), do které má výhradní přístup~\cite{dbperservice}.
   V takovém případě se samozřejmě může jednat i o fyzicky samostatné databázové servery.
\end{dl}

Oba přístupy mají svoje využití dle poskytovaných vlastností.



\subsection{Sdílená databáze}\label{subsec:shared-db}

Sdílená databáze poskytuje všechna svoje úložiště všem mikroslužbám, možná realizace je znázorněna na obrázku~\ref{fig:db-shared}. \g{ACID} vlastnosti zde jsou řízeny databází a logicky se ničím neliší od komunikace s monolitickou aplikací.

\begin{figure}[htbp]
   \centering
   \includegraphics[max width=\textwidth]{assets/db-shared}
   \caption{Architektura sdílené databáze}\label{fig:db-shared}
\end{figure}

\textbf{Výhody}
\begin{ul}
   \item Typické použití \g{ACID} pro provádění transakcí~\cite{shareddb}.
   \item Správa jedné databáze je jednodušší~\cite{shareddb}.
\end{ul}

\textbf{Nevýhody}
\begin{ul}
   \item Databázové migrace se netýkají pouze jedné mikroslužby, ale všech zároveň, musí se vyčlenit místo pro jejich uchovávání.
   \item V případě jakékoliv změny struktury databáze je třeba změnit a otestovat všechny mikroslužby~\cite{shareddb}.
   \item Jednotná databáze nemusí vyhovovat potřebám všech mikroslužeb~\cite{shareddb}.
\end{ul}



\subsection{Databáze pro každou službu}\label{subsec:msa-db-per-service}

V případě vyčlenění jednoho schématu, či celé databáze pro každou mikroslužbu, můžeme mluvit o druhém typu struktury využití relačních databází, možná realizace je znázorněna na obrázku~\ref{fig:db-per-micro}.


\textbf{Výhody}
\begin{ul}
   \item Zmenšení provázanosti mikroslužeb a dat, jež jsou využívány – lepší správa migrací a jiných změn~\cite{dbperservice}.
   \item Lepší možnosti škálování a volby databáze – mikroslužbě je možné poskytnout takové prostředí, které bude nejvhodnější~\cite{dbperservice}.
\end{ul}


\textbf{Nevýhody}
\begin{ul}
   \item Problém se spojováním dat z více úložišť – nelze například provádět \h{JOIN}, párovat dle cizího klíče s kontrolou integritního omezení~\cite{dbperservice}.
   \item Problém s transakčním zpracováním – transakční operace nad několika databázemi nelze jednoduše provádět a je lepší se jim úplně vyhýbat~\cite{dbperservice}.
   \item Komplikovaná správa a konfigurace většího počtu databází~\cite{dbperservice}.
\end{ul}


\begin{figure}[htbp]
   \centering
   \includegraphics[max width=\textwidth]{assets/db-per-micro}
   \caption{Architektura databází (schémat) pro každou službu}\label{fig:db-per-micro}
\end{figure}

Nevýhody databází rozdělených dle mikroslužeb lze řešit s pomocí některých návrhových vzorů, jsou ve stručnosti popsány v následujících podkapitolách.



\section{Spojování dat mezi mikroslužbami}\label{sec:msa-db-join}
Jako jeden z možných požadavků v rámci \g{MSA} je spojení dat mezi tabulkami, což v~rámci oddělených služeb znamená netriviální úkol.
Jako typický příklad lze vzít existenci tabulek \h{users} (osobní informace) a \h{payments} (složeno z částky a primárního klíče z \h{users}), jež se nachází v oddělených databázích, ze kterých je třeba získat seznam transakcí všech lidí s určitým věkem.
Spojení v tomto případě nemůžeme nechávat na databázi, situace se může řešit API kompozicí nebo \g{CQRS} přístupem.



\subsection{API kompozice}\label{subsec:msa-db-aggregate-api}

API kompozice je řešení bez využití dodatečných systémů.
Spočívá v logickém rozdělení původního dotazu na parciální dle jednotlivých mikroslužeb a následným provedení například simulace \h{JOIN} operace v aplikaci (v paměti)~\cite{apicomposition}.
Schématicky je tato operace znázorněna na obrázku~\ref{fig:api-composition}.
Samozřejmě takové řešení může přinést větší nároky na paměť v případě pracování obrovských datových celků.


\begin{figure}[htbp]
   \centering
   \includegraphics[max width=\textwidth]{assets/api-composition}
   \caption{Schéma vzoru API kompozice}\label{fig:api-composition}
\end{figure}



\subsection{CQRS}\label{subsec:msa-db-aggregate-cqrs}

\g{CQRS} vzor může sloužit jako řešení nedostatku paměti, případně výkonu, které vzniká operováním s daty v aplikaci.
Absence možnosti spojení je tu řešena dodatečnou, eventuálně konzistentní databází~\cite{msachris}, viz obrázek~\ref{fig:cqrs}.
Databáze slouží jako zdroj dat pro čtení, který obsahuje kopie původních dat a databázový pohled na strukturu potřebou pro provedení klientského dotazu.
Existující relační databáze mikroslužeb fungují jako persistentní úložiště pro zápis, uchovávají si informace a posílají potřebnou kopii dat do databáze pro čtení~\cite{cqrs}.
Takovým způsobem se vytváří úložiště, které obsahuje minimální potřebné množství informací například pro operaci \h{JOIN}.
Samozřejmě eventuální konzistenci takové databáze je nutné explicitně zajistit, aby nedocházelo ke ztrátě dat~\cite{cqrs}.



\begin{figure}[htbp]
   \centering
   \includegraphics[max width=\textwidth]{assets/cqrs}
   \caption{Zjednodušený model CQRS vzoru}\label{fig:cqrs}
\end{figure}



\section{Transakční zpracování zápisů}\label{sec:msa-db-transaction}
V relačních databázích jsou transakce chápány jako logické celky pro čtení nebo zápis informací, které dodržují \g{ACID} vlastnosti~\cite{dbtransactions}.
Pokud existuje víc datových zdrojů a~je třeba provést požadavek na zápis, jenž musí ovlivnit data v několika databázích, tak nastává problém neizolovanosti zápisu~\cite{msachris}.
Během zpracování požadavku každá mikroslužba na základě komunikace dostane příkaz o zápisu a provede je v souladu s \g{ACID}, pokud však alespoň jedna z dílčích částí nebude úspěšná, tak \h{ROLLBACK} transakce bude uskutečněna pouze v chybné.


Takové situace se mohou řešit například s pomocí distribuovaných transakcí, nebo návrhového vzoru \h{saga}~\cite{msachris}.

Distribuované transakce jsou metodou, která propojuje víc datových zdrojů (v tomto případě databází), zpravidla stejného typu, a během zpracování transakce zajišťuje \g{ACID} zápis napříč databázemi~\cite{distributedtransactions}.
Dá se říct, že distribuce odstraňuje jedné transakci omezení z hlediska počtu databází prostřednictvím vhodné konfigurace~\cite{distributedtransactionsintegration}.

Sága je realizována na straně serverové aplikace, jedná se o kontrolovanou sekvenci spouštění lokálních transakcí.
Každá dílčí transakce po svém provedení posílá zprávu nebo vyvolává událost, jež spouští další transakci dle určeného pořadí.
Pokud jedna z lokálních transakcí skončí chybou, pak obdobným způsobem se spouští sekvence zpětných transakcí všech předchozích úprav~\cite{saga}.
Tímto je docílena chybějící izolace napříč databázemi.
Příklad takové struktury lze vidět na obrázku~\ref{fig:saga}.



\begin{figure}[htbp]
   \centering
   \includegraphics[max width=\textwidth]{assets/saga}
   \caption{Koncept transakčního zpracování v ságách}\label{fig:saga}
\end{figure}
