\chapter{Analýza a implementace serverové části}\label{ch:server}


Vzhledem ke změněým požadavkům a potřeby změny architektury bylo rozhodnuto přepsat celou implementaci serverové části od začátku.
Původní zdrojový kód bude sloužit pouze jako inspirace pro realizaci některých funkcionalit.



\section{Databáze a migrace}\label{sec:server-db}



\section{Sestavování balíčku a kontejnerizace}\label{sec:server-compile}

Vzhledem k používaným technologiím zdrojový kód aplikací nelze spouštět bez předchozího zpracování.
TypeScript, jako samostatný jazyk, vyžaduje transpilaci kódu do JavaScript přes vhodný nástroj, například \h{tsc}.
Dle výchozího chování vytváří vedle každého rozpoznaného \h{.ts} souboru jeho transpilovanou kopii, importované a exportované části se neslučují.
V takovém stavu se \h{.js} soubory již dají spouštět za pomocí nástroje \h{node} (Node.js).

Obdobného chování, ale již bez explicitního transpilování, lze dosáhnout s pomocí nástroje \h{ts-node}.
Balíček \h{ts-node} je dle jednoho z autorů připraven pro produkční prostředí a program může být spuštěn bez dopadu na výkon~\cite{tsnodeprod}.
Tento výsledek by se dal považovat za uspokojivý, avšak ne v případě kontejnerizace.

Vzhledem k vnějším závislostem z adresáře \h{node\_modules}, které v obou případech zůstávají zachovány, bude nutné ponechávat celý adresář po \h{build} fázi v Docker obrazu.
To může mít za následek nevhodnou velikost výsledného kontejrenu, v daném projektu by to znamenalo přibližně \TODO{2GB} zbytečných dat (vývojářské nástroje, zdrojové kódy knihoven apod.) pro každou aplikaci.
V případě mohutnějších \g{MS} by dané číslo mohlo být mnohem větší a přidalo by negativní dopad na požadavky parametrů serverů.

Jako řešení se nabízí selektivní kopírování potřebných souborů z \h{node\_modules} adresáře do Docker obrazu.
Z tohoto důvodu do sestavování \g{MS} byl přidán nástroj \h{webpack} a další, potřebné pro vývoj, balíčky.
Tímto zásahem se přišlo o jednoduchost transpilace – \h{webpack} přebral úkol předzpracování \h{.ts} souborů, ale získalo se mnohem ohebnější prostředí.
Výstup vytvářený \h{webpack} je \h{.js} soubor, jenž nepotřebuje vnější závislosti a je soběstačny – zahrnuje v sobě kopie všech závislostí a potřebuje pouze vnější JavaScript interpet, který se získává z rodičovského Docker obrazu.

Lokální vývoj s tímto nastavením nepotřebuje žádnou logiku navíc.
Pro účely vývoje se přidal nástroj \h{nodemon}, jenž umožňuje hot-reloading\footnote{Hot-reloading je automatické přenačtení a spuštění programu, které se provádí po každé změně kódu.}.

Ve výsledku se každá \g{MS} začala sestavovat s pomocí \h{webpack}, výstupem je samostatná sada transpilovaných a minifikovaných souborů, jež je použita v druhé fázi sestavování.
Dockerfile konfigurace je detailně popsaná v kapitole \nameref{ch:deployment}.
