\chapter{Rozvoj systému}\label{ch:improvements}

\chaptersummary{
   \begin{ul}
      \item seznam bodů nutných pro dokončení \g{IS},
      \item seznam možných zdokonalení aktuální implementace.
   \end{ul}
}

Implementace daného \g{IS} v rámci této práce byla soustředěna kolem návrhu základních principů fungování architektury mikroslužeb a uplatnění zdokonaleného principu tvorby rozdělenéhé klientské aplikace.
Vzhledem k obrovskému rozsahu práce a nepředpokládané časové náročnosti vývoje v architektuře mikroslužeb, mnohé funkcionality, jež většinou představují monotónní implementaci \g{REST} rozhraní se zápisem do databáze dle určitých podmínek, nebyly realizováy.
Vše potřebné pro jejich vývoj však již bylo začleněno.

Tato kapitola definuje v sobě zbývající části systému, jenž mají být dodány vzhledem k původní analýze, a poskytuje seznam vylepšení aktuálního řešení.
K některým bodům je dodáno, jakým způsobem může být dané vylepšení realizováno.



\section{Neimplementované funkce}\label{sec:unimplemented}

\begin{dl}
   \item[Dořešení systému práv a zobrazování dat] – na základě specifikace v systému uživatelské role a nastavení projektu mají mít vliv na poskytované možnosti a dodávaná data.
   Rozdělení na role a možnost kontroly rolí byla realizována, avšak pouze na vysoké úrovni (volání rozhraní, zobrzování nebo skrytí projektu).
   Je nutné dodělat detailní kontrolu dat, které jsou poskytovány uživatelům na základě rolí a vytvořit výjimku pro administrátory.
   \item[Uživatelsky vhodné zobrazování chyb] – v rámci serverové části bylo vytvořeno relativně přehledné kaskádování chyb s propagací zpráv s přesně daným rozhraním, zdaleka ne všechny jsou však zpracovávány klientským rozhraním do uživatelsky příjemné podoby.
   \item[Doplnění vhodných uživatelských upozornění] – funkcionalita pro oznamování důležitých událostí je plně funkční, je nutné přidat tvorbu takových zpráv ve všech potřebných funkcích.
   \item[] .
   \item[] .
   \item[] .
\end{dl}

A další požadavky, označované od začátku jako rozvojové (viz kapitola 4):


\begin{dl}
   \item[Další možnosti vyhledávání projektů] – zadávání a vyhledávání projektů na základě unikátních značek (tagů).
   \item[Iterace a úkoly] – dělení projektů na iterace a pkoly, jenž by se daly plnit tvorbou obsahových částí projektu.
   \item[Snímky iterací] – odevzdávání stavu obsahu projektu, vzhledem k implementované funkcionalitě může být implementováno jako značka v gitu.
   Tzn.
   odevzdávat se bude historický bod projektu bez kopírování stavu projektu, jak to bylo realizováno v bakalářské práci.
   \item[Hodnocení iterací] – ohodnocení odevzdaného snímku a tudíž i iterace osobou, jenž je označována jako autorita v rámci \g{IS}.
   \item[Editace týmu] – editace projektových týmů ověřenými uživateli.
\end{dl}



\section{Možné zdokonalení systému}\label{sec:improvements}


\begin{dl}
   \item[Zlepšení logovacího systému] .
   \item[Aktivní monitoring fungování služeb] .
   \item[Monitoring a vyvažování zátěže] .
   \item[Zabránění volání interních přístupových bodů ze vně] .
   \item[Zakládání admin účtu] .
   \item[Nahrávání souborů] .
   \item[Zdokonalení testování] .
   \item[Optimalizace klienta] .
   \item[Zdokonalení vývoje interpretů obsahu] .
   \item[Paralelní úpravy obsahu] .
   \item[Separace kaskádových stylů a obsahu] .
   \item[UX/UI] .
   \item[] .

\end{dl}
