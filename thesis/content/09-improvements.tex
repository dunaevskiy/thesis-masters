\chapter{Rozvoj systému}\label{ch:improvements}

\chaptersummary{
   \begin{ul}
      \item seznam bodů nutných pro dokončení \g{IS},
      \item seznam možných zdokonalení aktuální implementace.
   \end{ul}
}

Implementace daného \g{IS} v rámci této práce byla soustředěna kolem návrhu základních principů fungování architektury mikroslužeb a uplatnění zdokonaleného principu tvorby rozdělenéhé klientské aplikace.
Vzhledem k obrovskému rozsahu práce a nepředpokládané časové náročnosti vývoje v architektuře mikroslužeb, mnohé funkcionality, jež většinou představují monotónní implementaci \g{REST} rozhraní se zápisem do databáze dle určitých podmínek, nebyly realizováy.
Vše potřebné pro jejich vývoj však již bylo začleněno.

Tato kapitola definuje v sobě zbývající části systému, jenž mají být dodány vzhledem k původní analýze, a poskytuje seznam vylepšení aktuálního řešení.
K některým bodům je dodáno, jakým způsobem může být dané vylepšení realizováno.

\newpage



\section{Nedokončená funkcionalita}\label{sec:unimplemented}

\begin{dl}
   \item[Dořešení systému práv a zobrazování dat] – na základě specifikace v systému uživatelské role a nastavení projektu mají mít vliv na poskytované možnosti a dodávaná data.
   Rozdělení na role a možnost kontroly rolí byla realizována, avšak pouze na vyšší úrovni (volání rozhraní, zobrzování, nebo skrytí projektu).
   Je nutné dodělat detailní kontrolu dat, které jsou poskytovány uživatelům na základě rolí, a vytvořit výjimku pro administrátory.
   \item[Uživatelsky vhodné zobrazování chyb] – v rámci serverové části bylo vytvořeno relativně přehledné kaskádování chyb s propagací zpráv s přesně daným rozhraním, zdaleka ne všechny jsou však zpracovávány klientským rozhraním do uživatelsky příjemné podoby.
   \item[Doplnění vhodných uživatelských upozornění] – funkcionalita pro oznamování důležitých událostí je plně funkční, je nutné přidat tvorbu takových zpráv ve všech potřebných funkcích.
   \item[Podpora jazykových verzí] – knihovna Next.js interně podporuje překlad rozhraní, nebylo však vůbec realizováno.
   Předpokládaný způsob integrace může být s využitím knihovny React-Intl.
\end{dl}

A další požadavky, označované od začátku jako rozvojové (viz kapitola \nameref{ch:spec}):


\begin{dl}
   \item[Další možnosti vyhledávání projektů] – zadávání a vyhledávání projektů na základě unikátních značek (tagů).
   \item[Iterace a úkoly] – dělení projektů na iterace a úkoly, jež by se daly plnit tvorbou obsahových částí projektu.
   \item[Snímky iterací] – odevzdávání stavu obsahu projektu, vzhledem k implementované funkcionalitě může být implementováno jako značka v gitu.
   Tzn.
   odevzdávat se bude historický bod projektu bez kopírování stavu projektu, jak to bylo realizováno v bakalářské práci.
   \item[Hodnocení iterací] – ohodnocení odevzdaného snímku a tudíž i iterace osobou, jenž je označována jako autorita v rámci \g{IS}.
   \item[Editace týmu] – editace projektových týmů ověřenými uživateli bez souhlasu uživatelů, včetně pozvánek apod.
\end{dl}


\newpage



\section{Možné zdokonalení systému}\label{sec:improvements}


\begin{dl}
   \item[Zlepšení logovacího systému] – aktuální řešení logování předpokládá v sobě záznam tvořený datumem, původem (název mikroslužby) a zprávou, která se předává přes brokera do NoSQL databáze.
   Poskytuje to možnost filtrování výsledků a vyhledávání, nemusí to být však odolné vůči výpadkům brokera zpráv.
   Potenciálně lepším řešením by mohlo být zavedení klasických, textových logů, které by mohly sloužit jako odlaďovací nástroj v případě zkoumání mikroslužby bez prostředí.
   Zároveň je možné slepšit strukturu samotného logu.
   \item[Aktivní monitoring fungování služeb] – v klientské aplikaci aktuálně existuje stránka v administrátorském panelu, která \h{pull} způsobem zjištuje aktuální stav mikroslužeb.
   Tato funkcionalita by se mohla vyčlenit do samostatně fungujícího sledování s upozorněním, například přes e-mail, že část systému měla výpadek.
   Případně prozkoumat a využít již existující řešení.
   \item[Monitoring a vyvažování zátěže] – se sledováním stavu mikroslužeb je rovněž spojeno sledování vytíženosti služeb a brkeru jako takového.
   Dle zběžného průkumu vyplynulo, že by se dala používat kombinace \h{Prometheus} pro sledování a \h{Grafana} pro vizualizaci, nicméně vyžadovalo by to komplexnější integraci.
   \item[Kontrola kompatibility mikroslužeb] – jednotlivé mikroslužby mohou být aktuálně verzovány dle jejich rozhraní.
   Není však dořešena vzájemná kompatibilita na základě závislostí.
   Potenciálním řešením by mohla být definice závislostí v rámci sémantického verzování a kontrola kompatibility v monitorovací mikroslužbě.
   \item[Zabránění volání interních přístupových bodů ze vně] – Gateway v současné době poskytuje ven všechna \g{REST} rozhraní (externí i interní) s tím, že interní kvůli provolání vyžadují speciální autorizační hlavičku s klíčem.
   V ideálním případě by se však interní rozhraní nemělo vůbec vystavovat.
   Řešením by mohla být explicitní filtrace požadavků na straně \h{nginx} nebo oddělení interního rozhraní s pomocí \g{URI} a překonfigurování všech mikroslužeb na nový typ zdroje dat.
   \item[Zakládání admin účtu] – administrátorský účet je nyní součástí základní sady dat pro databázi.
   Může se nahradit samostatnou, jernorázovou počáteční registrací administrátorského účtu před prvním startem \g{IS}.
   \item[Nahrávání souborů] – v rámci tvorby obsahu může vzniknout potřeba nahrávat soubory, může to být řešeno s pomocí interpretátoru obsahu.
   \item[Zdokonalení testování] – aktuální testování počítá s otestováním samostatných funkcí s pomocí jednotkových testů a celistvost fungování serverové části s pomocí integračních.
   V případě dokončení uživatelského rozhraní by mohlo vzniknout testování na základě simulace uživatelské činnosti a další testování, například zátěže, bezpečnosti apod.
   \item[Zdokonalení vývoje interpretů obsahu] – interprety pro obsahy nyní jsou tvořeny zčásti manuální činností, která může být zautomatizována.
   Vývoj interpretů by následně mohl probíhat jako vývoj běžné React apliakce, kde by během sestanovávní programu redukoval obsah na nezbytené minimum.
   \item[Rozšíření konceptu mikroslužeb u klienta] – stejně, jako jsou realizovány interprety obsahu, by mohl fungovat zbytek klientské aplikace.
   Zde je však otázka, zda by se to v případě relativně malého projektu vyplatilo.
   \item[Paralelní úpravy obsahu] – úpravy stejné části obsahu jsou nyní zabezpečeny s pomocí kontroly \h{hash} klíče původního obsahu.
   Mohlo by být vhodné realizovat paralelní úpravu 1 části více uživateli.
   \item[Separace kaskádových stylů a obsahu] – v současné implementaci kvůli vlastnostem \g{CSS} není dokonale řešeno ovlivňování jednotlivých částí obsahu existujícími globálními styly.
   Měl by se ideálně definovat jednotný styl pro základní prvky (též nazývaný \enquote{style-guide}), který by ovlivňoval i obsahy, ale zbytek by mohl být napsaný s pomocí technologie \g{CSS} modulů.
   Tím by se vyřešily potenciální konflikty mezi \g{CSS} pravidly.
   \item[UX/UI] – rozhraní může být přizpůsobeno mobilnímu rozhraní a vizuálně odladěno.
\end{dl}

Obecně se dá říct (z osobních zkušeností vzhledem k předchozí implementaci \g{IS}), že architektura mikroslužeb má řadově víc možností, kde se dá zdokonalit, v porovnání s monolitickou architekturou.
Taková volnost v řešení různých aspektů na jednu stranu přináší větší intuitivní chápání, na druhou zbytečně komplikuje vývoj a nevhodně oddaluje dosažení samotného cíle alespon v první, konzistentní podobě.
Práce s plnohodnotným dokončením tohoto \g{IS} se pro jednoho člověka může protáhnout na nevhodně dlouhou dobu.
