\chapter{Srovnání výsledků}\label{ch:compare}

\chaptersummary{
   \begin{ul}
      \item zhodnocení výhody přechodu z \g{MA} na \g{MSA},
      \item seznam zásadních změn ve srování se starou implementací \g{IS}.
   \end{ul}
}

Informační systém pro správu projektů byl úspěšně převeden na architekturu mikroslužeb.
Tento přechod jednoznačně přispěl k systematizaci celého systému a začal se vnímat jako mnohem uspořádanější, než v monolitické architektuře.
Jsou vidět striktně vymezené hranice a chování, systém je mnohem ohebnější z hlediska implementace a rozvojových možností.
Mikroslužby však výrazně snížily rychost vývoje a prodloužily fázi potřebné analýzy.
Bylo nutné řešit víc problémů, se kterými se u monolitické architektury člověk nesetká – zejména infrastrukturní komunikaci a rozdělení \g{API} mezi různými službami.
Nepomáhala tomu ani rozmanitost možností návrhu architektury jako takové.
To vše mělo za následek zmenšení rozsahu dodávané funkcionality, oproti původnímu, monolitickému řešení, i když času, věnovaného implementaci, bylo řádově více.

Realizace \g{IS} v architektuře mikroslužeb byla, dle subjektivního hlediska, obecně přínosná, avšak ne dokonalá.
Je možné, že jiný typ dekompozice a volby komunikace v tomto případě by mohl zredukovat potřebné prostředky pro realizaci a ušetřit čas.
Prozkoumat všechny kombinace v rámci této diplomové bohužel nebylo uskutečnitelné.

\newpage

V následujícím přehledu se uvádí seznam zásadních změn, které nastaly ve srovnání se starou implementací:

\begin{ul}
   \item Původní \g{MA} architektura potřebovala mnohem méně časů a prostředků pro vývoj.
   To se nejspíš bude projevovat i v pozdějším rozvoji.
   \item Vnímání fungování nové serverové architektury je víc intuitivní, než u předchozí.
   Přispěla tomu zejména dekompozice.
   \item Mikroslužby z tohoto systému lze ještě víc zobecnit a používat v jiných projektech.
   V případě monolitu by taková možnost neexistovala.
   \item Zhoršila se organizace vývoje pro menší počet vývojářů.
   Správa několika mikroslužeb jednim člověkem je méně přehledná, než správa monolitu.
   \item Kvůli zvolené struktuře začalo docházet ke kontrolované duplicitě kódu mezi repozitáři.
   \item Nový způsob implementace interpretů obsahů v klientské části přinesl jednodušší vývoj a pohodlnější prostředí, i když zatím omezil technologii pouze na React.
   \item Původní systém obsahoval víc funkcionality s menším důrazem na přehlednost architektury, novější prioritizuje architekturu a aktuálně zaostává ve funkcionalitě.
   \item Testování v nové realizaci je automatizováno a vyžaduje méně manuálních zásahů.
   \item Došlo k úplné kontejnerizaci systému.
\end{ul}





