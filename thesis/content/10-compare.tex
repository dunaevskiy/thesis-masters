\chapter{Srovnání výsledků}\label{ch:compare}

\chaptersummary{
   \begin{ul}
      \item zhodnocení výhody přechodu \g{MA} na \g{MSA},
      \item seznam subjektivních tvrzení a rad o vývoji mikroslužeb.
   \end{ul}
}

Informační systém pro správu projektů byl úspěšně zmigrovaný na architekturu mikroslužeb.
Tento přechod jednoznačně přispěl k systematizaci celého systému a začal se vnímat jako mnohem uspořádanější, než v monolitické architektuře.
Jsou vidět striktně vymezené hranice a chování, systém je mnohem ohebnější z hlediska implementace a rozvojových možností.
Mikroslužby však výrazně snížily rychost vývoje a prodloužily fázi potřebné analýzy.
Bylo nutné řešit víc problémů, se kterými se u monolitické architektury člověk nesetká – zejména infrastrukturní komunikaci a rozdělení \g{API} mezi různými službami.
Nepomáhala tomu ani nepřesnost architektury jako takové.
To vše mělo za následek zmenšení oblastni dodávané funkcionality, oproti původnímu, monolitickému řešení, i když času, věnovaného implementaci, bylo řádově víc.

Realizace \g{IS} v architektuře mikroslužeb byla, dle subjektivního hlediska, obecně přínosná, avšak ne dokonalá.
Je možné, že jiný typ dekompozice v tomto případě by mohl zredukovat potřebné prostředky pro realizaci a ušetřit čas.


V následujícím přehledu se uvádí seznam subjektivních tvrzení, na které se přišlo po prozkoumání \g{MSA} a dokončení migrace \g{IS}:

\begin{ul}
   \item Existuje intuitivní chápání návrhu \g{MSA}, ale ve srovnání s \g{MA} potřebuje více času a více prostředků.
   \item \g{MSA} je vhodné pro mnohonásobně složitější systémy, kde by výhoda mohla být vidět v pozdějích vývojářských fázích.
   \item \g{MSA} se nejspíš nehodí pro vývoj v menším vývojářském týmu.
   \item Klietská realizace mikroslužeb je špatně dosažitelná a méně prozkoumaná.
   \item Testování mikroslužeb v jednoduchých případech se ničím neliší od testování \g{MA}, avšak v těch komplexnějších vyžaduje další čas a prostředky.
   \item Existující mikroslužby jsou pohodlně použitelné na více projektech, aniž by musely být pozměněny.
   \item Je výhodné rozdělovat specifikaci dle mikroslužeb tak, aby se nemusely řešit transakce napříč systémy.
   \item Mikroslužby budou mezi sebou sdílet společnou část impelmentace, je třeba rozumně vyřešit způsob tohoto sdílení před samotným vývojem.
\end{ul}





