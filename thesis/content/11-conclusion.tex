Tato diplomová práce byla věnována analýze architektury mikrozlužeb a návrhu realizace přechodu existujícího prototypu informačního systému pro správu projektů z monolitické architektury na architekturu mikroslužeb.
V souladu se zadáním bylo stanovenu několik cílů, z nichž všechny byly úspěšně dokončeny.

V rámci práce na základě existujících podkladů byla nejdřív provedena analýza poskytovaného systému.
Výstupem této činnosti se stal seznam negativních a pozitivních skutečností, který spolu s aktualizovanou rešerší fakultních systémů se stal předlohou pro novou specifikaci projektu.
Následně proběhla analýza architektury mikroslužeb, jenž byla vzhledem k obrovskému rozsahu a nejednoznačným přístupům zredukována na potřebné minimum pro vytvoření nového prototypu.
Samostatná kapitola byla věnována využití databází a transakčnímu zpracování v případě oddělených částí relačních databází.

S využitím všech získaných znalostí byl navržen a realizován přechod serverové a klientské části aplikace na architekturu mikroslužeb.
Nejvíc přínosné nebo implementačně zajímavé aspekty byly popsány v rámci kapitol věnujících se vývoji aplikace.
Serverová část byla rozdělena na mikroslužby s choreografií, propojena s Github, který nahradil dřívější MongoDB, a plně kontejnerizována s pomocí Docker.
Kromě databází pro ukládání persistentních dat se začal využívat broker zpráv RabbitMQ\@.
Klientská část byla rovněž kontejnerizována a začala pouzívat novou funkcionalitu pro vykreslování částí projektů s pomocí vnějších React komponent distibuovaných v podobě \g{ES6} modulů.

Manuální testování bylo do určité míry nahrazeno automatizovaným.
Statická analýza kódu a jednotkové testy se staly součástí životního cyklu vývoje aplikace a automaticky se spouštějí v potřebných chvílích.
Funkčně-integrační testy je třeba pouštět manuálně, byly definovány v samostatném repozitáři a automaticky testují poskytované rozhraní z hlediska přístupových práv a typických testovacích scénářů.

Pro výsledný stav informačního systému byl na konec definován možný budoucí rozvoj a sepsána vývojářská dokumentace.
Celý výsledek byl porovnán s původním řešením – změna byla přínosná z hlediska architektury, přinesla však zvýšené náklady pro vývoj, což pro tak malý projekt nemusí být výhodné.

Zdrojové kody všech částí \g{IS} byly zveřejněny na Github serveru.
Spolu s dodatečnými materiály se rovněž nachází na přiloženém médiu.
Projekt, jako celek, není v dokončené fázi a o rozvoji jeho současného stavu se bude uvažovat i bez akademických cílů.

Architektura mikroslužeb má svůj účel a využití v rámci vývoje systémů, je intuitivnější, než některé jiné architektury, ale její realizace je komplikovanější, než se na první pohled zdá.
Může najít svoje uplatnění v komplexních strukturách a systémech s potřebou parciálního škálování.
